\documentclass{datateknologsektionen-document}

% Command for info@d-sektionen.se
\newcommand*{\infomail}{\href{mailto://info@d-sektionen.se}{info@d-sektionen.se}}

\title{Policy kring datahantering}
\date{2018-05-24}

\begin{document}
\hspace{0pt}
\vfill
\begin{center}
  \Huge\textbf{Policy kring datahantering}

  \huge Datateknologsektionen

  \large
  Organisationsnummer: 822002-1409

  Version: 1.0

\end{center}
\vfill
\hspace{0pt}
\pagebreak

% \section*{Versionshistorik}
% \begin{footnotesize}
%   \begin{tblr} to \linewidth { |l|l|X[2]|X| }
%     \hline
%     Version & Datum & Anmärkning & Ansvarig \\ \hline
%     0.1 & 2015-02-01 & Första utkastet & Robin Abrahamsson \\ \hline
%     0.2 & 2015-03-20 & Lagt till versionshistorik, innehållsförteckning och en del förtydliganden & Robin Abrahamsson, Simon Lindén \\ \hline
%     0.3 & 2015-04-23 & Ytterligare förtydligande och omstrukturering & Robin Abrahamsson, Simon Lindén, Jonathan Nyman \\ \hline
%     0.4 & 2017-11-23 & Lade till \S 2.5 som syftar på den interna köp- och säljgruppen. Flyttade gamla \S 2.5 till \S 2.6. & Daniel Roos \\ \hline
%   \end{tblr}
% \end{footnotesize}

% \pagebreak

\tableofcontents

\pagebreak

\section{Formalia}
\subsection{Sammanfattning}
Denna policy beskriver hur D-sektionen hanterar personuppgifter, i vilka fall samt vilka personuppgifter som hämtas in, hur de lagras samt hur länge de sparas i sektionen.
Policyn definierar även hur samtycke bör inhämtas.
För frågor om behandling av personuppgifter, var vänlig vänd dig till \infomail{}.

\subsection{Syfte}
Syftet med denna policy är att beskriva hur D-sektionen behandlar personuppgifter för att säkerställa en tydlighet både internt och externt.
Syftet med arbetet kring personuppgiftshantering handlar om att säkerställa berörda parters rätt till sin personliga integritet.

\subsection{Omfattning}
Denna policy gäller för samtliga utskott eller andra grupper som representerar D-sektionen som hanterar någon form av personuppgifter.
Policyn skall även användas för att externt kommunicera hur D-sektionen hanterar personuppgifter som kommer in till, eller behandlas av, sektionen.

\section{Rättigheter}
Du som har information lagrad i våra system har följande rättigheter:
\begin{itemize}
  \item \textbf{Rätt till tillgång.} Du kan när som helst maila \infomail{} för att begära att få reda på vilka personuppgifter som sektionen lagrar om dig.
  \item \textbf{Rätt till rättelse.} Du kan när som helst maila \infomail{} för att begära ändring av felaktiga personuppgifter som sektionen lagrar om dig.
  \item \textbf{Rätt till radering.} Du kan när som helst maila \infomail{} för att ta bort personuppgifter som sektionen lagrar om dig om något av följande uppfylls:
        \begin{itemize}
          \item Uppgifterna inte längre är nödvändiga för de ändamål för vilka de har samlats in eller behandlats.
          \item Om behandlingen grundar sig enbart på ditt samtycke och du återkallar detta.
          \item Du invänder mot en intresseavvägning D-sektionen har gjort baserat på berättigat intresse och ditt skäl för invändning väger tyngre än D-sektionens berättigade intresse.
          \item Personuppgifterna behandlas på ett olagligt sätt.
          \item Personuppgifterna måste raderas för att uppfylla en rättslig förpliktelse som D-sektionen omfattas av.
        \end{itemize}
        Observera att D-sektionen har rätt att neka en begäran av borttagning om annan legal skyldighet finns.
  \item \textbf{Rätt till invändning.} I de fall D-sektionen använder en intresseavvägning som laglig grund för ett ändamål har du möjlighet att invända mot behandlingen. För att kunna fortsätta behandla dina personuppgifter efter en sådan invändning behöver D-sektionen kunna visa ett berättigande skäl för den aktuella behandlingen som väger tyngre än dina intressen, rättigheter eller friheter.
  \item \textbf{Rätt till begränsning.} Du har rätt att begära en tillfällig begränsning av behandling av din data exempelvis om du anser att datan inhämtats felaktigt, att datan inte är korrekt eller att behandlingen kränker dina rättigheter eller friheter. När du invänt mot behandling av din data får D-sektionen fortsätta behandla din data under den tid som kontrollen pågår. Om behandlingen av dina personuppgifter begränsas tillfälligt, kommer D-sektionen underrätta de parter som vi har lämnat ut uppgifterna till, om att denna tillfälliga begränsning har ägt/äger rum. Ovanstående gäller ej i de fall det skulle visa sig omöjligt eller innebära en alltför betungande insats.
  \item \textbf{Anmälan av överträdelse.} Om du anser att dina personuppgifter behandlas i strid med gällande regelverk bör du anmäla det till oss snarast. Du kan också lämna klagomål till Datainspektionen.
\end{itemize}

\section{Hantering av data}
Nedan beskrivs de instanser där D-sektionen kan samla in någon av dina personuppgifter. Har
du några frågor får du gärna kontakta \infomail{}.

\subsection{Register över medlemmar}
Alla medlemmar i D-sektionen som någon gång loggat in på sektionens hemsida finns lagrade i dess databas.
Ett liu-id behöver lagras för att kunna identifiera dig som medlem vid framtida inloggning.
Förnamn och efternamn används till till exempel en användares profil eller dess svar i formulär.
Som medlem har du rätt till att begära borttagning av dessa uppgifter mot att sektionen ej längre kan garantera din inloggning till hemsidan.

\begin{tblr}{l X}
  \textbf{Uppgift(er):}    & Förnamn, efternamn, liu-id. \\
  \textbf{Rättslig grund:} & \textit{Intresseavvägning}. \\
  \textbf{Lagringsperiod:} & Medlemskapets period.
\end{tblr}

\subsection{Veckomail}
För att kunna skicka ut nyheter till sektionens medlemmar finns det för sektionens informationsansvarige en möjlighet att skicka ut veckomail.
Varje år läggs alla nyantagna till i denna maillista.

\begin{tblr}{l X}
  \textbf{Uppgift(er):}    & Epostadress. \\
  \textbf{Rättslig grund:} & \textit{Samtycke}. Veckomailet är helt frivilligt, och du som registrerad kan gå ur maillistan själva genom en länk som skickas med i varje mail. \\
  \textbf{Lagringsperiod:} & Medlemskapets period.
\end{tblr}

\subsection{Register över sektionsaktiva}
Information om samtliga sektionsaktiva inom D-sektionen finns lagrade.
Denna information är ämnat att användas endast för sektionsrelaterad verksamhet av sektionens kanslimedlemmar, som till exempel för att ha möjlighet att ta kontakt med personen vid behov.

\begin{tblr}{l X}
  \textbf{Uppgift(er):}    & Liu-id, förnamn, efternamn, samtliga utskottsmedlemskaper, poster i dessa utskott, epostadress, telefonnummer. \\
  \textbf{Rättslig grund:} & \textit{Intresseavvägning}. När sektionsaktiva uppmanas lämna dessa uppgifter ska uppgifternas användningsområden göras klart för denna. \\
  \textbf{Lagringsperiod:} & Det verksamhetsår som uppdraget gäller.
\end{tblr}

\subsection{Ansökan sektionsaktiv}
Information om samtliga ansökningar till utskott eller annan grupp som representerar D-sektionen finns lagrade.
Denna information är ämnat att användas endast för sektionsrelaterad verksamhet som till exempel för att föra statistik eller för att ha möjlighet att ta kontakt med personen vid behov.

\begin{tblr}{l X}
  \textbf{Uppgift(er):}    & Förnamn, efternamn, epostadress, telefonnummer. \\
  \textbf{Rättslig grund:} & \textit{Intresseavvägning}. I varje ansökningsformulär ska utskottet eller annan grupp som representerar D-sektionen klart beskriva hur personuppgifterna som anges i formuläret kommer att användas. \\
  \textbf{Lagringsperiod:} & Tills det att ansökan är färdigbehandlad.
\end{tblr}

\subsection{Epost}
Under ett verksamhetsår kommunicerar sektionens representanter ibland via epost.

\begin{tblr}{l X}
  \textbf{Uppgift(er):}    & De personuppgifter som framgår i epost-meddelanden, inklusive de epost-adresser som meddelandet skickas till.                                                                                                                                    \\
  \textbf{Rättslig grund:} & \textit{Intresseavvägning}. \\
  \textbf{Lagringsperiod:} & Tills det att ärendet som personuppgifterna gäller inte längre gör sektionen nytta.
\end{tblr}

\subsection{Sektionens röstningssystem}
Under sektionens möten används ibland ett elektroniskt rösningssystem för att underlätta vid omröstningar av olika slag.

\begin{tblr}{l X}
  \textbf{Uppgift(er):}    & Liu-id.\\
  \textbf{Rättslig grund:} & \textit{Konkludent avtal}. För att kunna se till att en person inte kan rösta mer än en gång per omröstning måste en lista över personer som röstat kunna tas fram för en given omröstning. \\
  \textbf{Lagringsperiod:} & Tillsvidare. Databasen över användare rensas ej i syfte att kunna ge speciella roller till användare.
\end{tblr}

\subsection{Aktivitetsrelaterad information}
Till de aktiviteter som anordnas av ett utskott eller annan grupp som representerar D-sektionen kan det hända att det samlas vissa personuppgifter såsom specialkost (om förtäring erbjuds), kontouppgifter (vid inbetalning av eventuell avgift), namn, kontaktuppgifter, bilder från aktiviteten.

\begin{tblr}{l X}
  \textbf{Uppgift(er):}    & De uppgifter som förfrågas i aktivitetens anmälningsformulär. \\
  \textbf{Rättslig grund:} & \textit{Intresseavvägning}. I varje anmälningsformulär ska utskottet eller annan grupp som representerar D-sektionen tydligt beskriva hur personuppgifterna som anges i formuläret kommer att användas, samt vilka personuppgifter som kan komma att samlas in under aktivitetens gång, till exempel bilder. \\
  \textbf{Lagringsperiod:} & Under tiden uppgifterna är relevanta för genomförandet för eventet samt tills dess att övriga nödvändiga behandlingar, exempelvis betalning, har fullgjorts. Eventuella bokföringsunderlag som uppkommer i samband med betalning för event sparas enligt Bokföringslagen. Lagringsperioden för eventuella bilder kan lagras tillsvidare i syfte av historisk bevaring.
\end{tblr}

\subsection{Räkning till Datateknologsektionen}
När du har en utgift du vill ha ersättning för via D-sektionen fyller du i en blankett som heter ``Räkning till Datateknologsektionen'' för att vår kassör ska kunna återbetala dig.

\begin{tblr}{l X}
  \textbf{Uppgift(er):}    & Clearing-nr, konto-nr, bank, förnamn, efternamn, underskrift. \\
  \textbf{Rättslig grund:} & \textit{Avtal}. Formuläret \textit{Räkning till Datateknologsektionen} ber om godkännande för lagring av den information som i formuläret ifyllts. \\
  \textbf{Lagringsperiod:} & Fram till och med det sjunde året efter utgången av det kalenderår då räkenskapsåret avslutades, enligt bokföringslagen.
\end{tblr}

\subsection{Kassörers personnummer till bank}
Sektionens kassör samlar in personuppgifter på nyinvalda kassörer under sektionen i syfte att skicka vidare till banken.

\begin{tblr}{l X}
  \textbf{Uppgift(er):}    & Förnamn, efternamn, personnummer. \\
  \textbf{Rättslig grund:} & \textit{Intresseavvägning}. \\
  \textbf{Lagringsperiod:} & Tills det att uppgifterna lämnats till banken.
\end{tblr}

\subsection{Nyckelkontrakt}
Ett nyckelkontrakt för Kårallen-nycklar finns i ett av sektionsrummen i syfte att överlåta ansvar för nyckeln och dess användning till personen i fråga.

\begin{tblr}{l X}
  \textbf{Uppgift(er):}    & Förnamn, efternamn, personnummer, telefonnummer, liu-id, relevant utskottsmedlemskap, post i relevant utskott, verksamhetsår. \\
  \textbf{Rättslig grund:} & \textit{Intresseavvägning}. Sektionen har ett starkt intresse av att ha kvar dessa personuppgifter eftersom dess nycklar är på vift.\\
  \textbf{Lagringsperiod:} & Tills det att nyckeln är återlämnad.
\end{tblr}

\subsection{Mötesprotokoll}
Under sektionens och styrelsens möten förs av mötets sekreterare ett protokoll som senare är tillgängligt för sektionens alla medlemmar via sektionens hemsida.

\begin{tblr}{l X}
  \textbf{Uppgift(er):}    & Förnamn, efternamn, personnummer. \\
  \textbf{Rättslig grund:} & \textit{Allmänt intresse}. Det ligger i organisationens och medlemmarnas intresse att
  möjliggöra transparens i sektionen.\\
  \textbf{Lagringsperiod:} & Tillsvidare.
\end{tblr}

\subsection{Övriga protokoll}
Under ett verksamhetsår förekommer personuppgifter i diverse former naturligt i protokoll.
Protokoll som förts av ett av sektionens utskott eller annan grupp som representerar sektionen
ska i största möjliga mån exkludera personuppgifter.

\begin{tblr}{l X}
  \textbf{Uppgift(er):}    & Förnamn, efternamn, personnummer, eller annan personuppgift som av det då pågående mötet bestämts ska inkluderas. \\
  \textbf{Rättslig grund:} & \textit{Intresseavvägning}. Övriga protokoll bör försöka minimera användandet av personuppgifter så mycket som går för att uppfylla kravet för den rättsliga grunden intresseavvägning.\\
  \textbf{Lagringsperiod:} & Tills utskottets verksamhet avslutats. Efter det ska protokollen antingen helt tas bort eller censureras på personuppgifter.
\end{tblr}

\subsection{Sektionsfotografering}
D-sektionens årliga sektionsfotografering sker med syfte av historisk bevaring.

\begin{tblr}{l X}
  \textbf{Uppgift(er):}    & Bild. \\
  \textbf{Rättslig grund:} & \textit{Konkludent avtal}. Detta genom att delta i fotograferingen.\\
  \textbf{Lagringsperiod:} & Tillsvidare.
\end{tblr}

\subsection{Nyckellista}
När en medlem eller annan använder någon av sektionens nycklar fyller denne i ett formulär som ligger precis bredvid nycklarna. Dessa går till t.ex. sektionens bil eller förråd.

\begin{tblr}{l X}
  \textbf{Uppgift(er):}    & Förnamn, efternamn, telefonnummer. \\
  \textbf{Rättslig grund:} & \textit{Konkludent avtal}. Genom att låna en av sektionens nycklar är det en självklarhet att ditt namn och telefonnummer dokumenteras så att du kan kontaktas ifall att nycklarna behövs omgående.\\
  \textbf{Lagringsperiod:} & Tills pappret med uppgifter är fyllt och ärendena är avslutande.
\end{tblr}

\subsection{Bil-logg}
När en medlem eller annan hyr sektionens bil samlas kontaktuppgifter.

\begin{tblr}{l X}
  \textbf{Uppgift(er):}    & Liu-id, relevant utskottsmedlemskap. \\
  \textbf{Rättslig grund:} & \textit{Avtal}. I sektionens bilavtal beskrivs hur de insamlade personuppgifterna
  kommer användas.\\
  \textbf{Lagringsperiod:} & Tillsvidare i bokföringssyfte.
\end{tblr}

\end{document}