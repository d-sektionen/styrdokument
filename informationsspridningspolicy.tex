\documentclass{datateknologsektionen-document}

\title{Policy för informationsspridning}
\date{2017-11-23}

\begin{document}
\hspace{0pt}
\vfill
\begin{center}
  \Huge\textbf{Policy för informationsspridning}

  \huge Datateknologsektionen

  \large
  Organisationsnummer: 822002-1409

  Version: 0.4

\end{center}
\vfill
\hspace{0pt}
\pagebreak

\section*{Versionshistorik}
\begin{footnotesize}
  \begin{tblr}{
    width = \linewidth,
    colspec = { |l|l|X[2]|X| }
  }
    \hline
    Version & Datum & Anmärkning & Ansvarig \\ \hline
    0.1 & 2015-02-01 & Första utkastet & Robin Abrahamsson \\ \hline
    0.2 & 2015-03-20 & Lagt till versionshistorik, innehållsförteckning och en del förtydliganden & Robin Abrahamsson, Simon Lindén \\ \hline
    0.3 & 2015-04-23 & Ytterligare förtydligande och omstrukturering & Robin Abrahamsson, Simon Lindén, Jonathan Nyman \\ \hline
    0.4 & 2017-11-23 & Lade till \S 2.5 som syftar på den interna köp- och säljgruppen. Flyttade gamla \S 2.5 till \S 2.6. & Daniel Roos \\ \hline
  \end{tblr}
\end{footnotesize}

\pagebreak

\tableofcontents

\pagebreak

\section{Inledning}
Det här är ett dokument som behandlar informationsspridningen i Datateknologsektionens
kommunikationskanaler. Det som står i detta dokument ska ses som riktlinjer för hur olika
typer av information ska hanteras. Om behov finns ska detta dokument kunna uppdateras vid
behov för att spegla rådande förhållande. Finns det oklarheter ska man försöka se till
studentens intressen i första hand.
\subsection{Bakgrund}
Datateknologsektionens olika nyhetsflöden är en attraktiv plats att sprida information på till
sektionens medlemmar. Eftersom det kan vara svårt att komma i kontakt med dessa personer
på annat sätt så finns det företag som är beredda att betala pengar för att dela information i
dessa flöden. Det har gjort det svårt att hitta en balans på vem som måste betala och vilken
information som är mest intressant och viktigast för sektionens medlemmar.
\subsection{Syfte}
Syftet med detta dokument är att se till att sektionens medlemmar får ett intressant
informationsflöde. Tanken är att detta dokument ska användas för att alla inom sektionen ska
ha samma bild av hur informationsspridningen sker och samtidigt se till så att
informationsspridningen sker på ett liknande sätt även i framtiden.
\section{Informationskanaler och dess användning}
Primärt ska man försöka använda sektionens kalender och hemsida för att dela information.
\subsection{Sektionens kalender}
Sektionens kalender är digital och ska finnas på hemsidan med möjlighet för vem som helst
att prenumerera på. Den kan användas till alla möjliga händelser och alla utskottsordföranden
ska kunna skapa händelser och se till att informationen som ska förmedlas är tydlig och
välskriven. Det är endast utskottsorföranden som får tillgång att ändra i kalendern.
\subsection{Hemsidan}
På hemsidan ska det finnas en del olika flöden där sektionsnyheter ska synas mest. De olika
flödena som ska finnas är
\begin{itemize}
  \item Sektionsnyheter
  \item Nyheter från LiU, LinTek och studentföreningar
  \item Företagsevent
  \item Platsannonser
  \item Examensarbeten
\end{itemize}

\subsection{Facebooksidan Datateknologsektionen}
Denna punkt avser sidan på Facebook som heter Datateknologsektionen och som medlemmar
på Facebook kan “Gilla”. Denna sida ska främst användas som portal till sektionshemsidan
och man ska därmed dela nyheter som sedan tidigare är publicerade på hemsidan. Det kan
också finnas tillfällen då annan information är i fokus men då kommer det märkas på det som
delas.
\subsection{Facebookgruppen D-sektionen}
Denna punkt avser Facebookgruppen vid namn D-sektionen. Denna grupp ska användas för
att dela nyheter från sektionshemsidan men ska vara öppen för resten av sektionen till att dela
diverse nyheter, event, prylar att sälja m.m. som relaterar till utbildningen och kan vara av
intresse för medlemmarna. Gruppadministratörer ska vara minst sektionens ordförande och
infochef. Denna sida behöver administreras och underhållas av gruppadministratörerna med
hjälp av Näringslivsutskottet (NärU) som sköter företagskontakter. Gruppen är till för
sektionens medlemmar och ska därför användas för att nyttja deras intressen.
\subsection{Facebookgruppen för försäljning}
Denna punkt avser Facebookgruppen vid namn D-sektionens köp och sälj. Denna grupp ska
primärt användas för försäljning av kursmaterial som t.ex. kursböcker, exempelhäften och
formelsamlingar, men kan även användas för försäljning av annat. Gruppadministratörer ska
vara minst sektionens infochef. Gruppen ska helt sakna näringslivsrelaterad anknytning, och
ska modereras hårt av gruppadministratörerna för att upprätthålla detta.
\subsection{Infomail}
Allmänt ska viktig information som postas på hemsidan skickas med i infomailet. Övrig
information som skickas med i infomailet är sådant som mailats till Infochefen och som i
samarbete med utskottsordföranden är av intresse för sektionens medlemmar.
\section{Informationstyper och dess hantering}
\subsection{Sektionsinformation}
Ska ha högst prioritet att visas. Den ska i första hand postas på hemsidan i form av nyheter
och i kalendern i form av händelser. Information som inkluderas här är information om
sektionsmöten, styrelsemöten, D-Pubar, sektionsfotboll/-innebandy och annat som kan vara
av intresse för sektionens medlemmar som sektionsaktiva arrangerar.
\subsection{LinTek/LiU/Studentföreningar}
Information från LinTek och Linköpings universitet (LiU) ska finnas med i samma
informationsflöde som studentföreningar. Den filtreringen som ska ske är att man ska se till
att den information som postas är relevant för sektionens medlemmar. Om den information
som vill delas är högaktuell och anses bör nå så många som möjligt är det godkänt att även
dela denna information i infomail och på Facebook.
\subsection{Företag}
Denna paragraf rör främst hur NärU ska arbeta för att dela information från företag.
\subsubsection{Platsannonser/Sommarjobb/Företagsevent}
Dessa typer av annonser ska gå genom NärU då detta är någonting man ska betala pengar för
att man som företag ska få visa det för sektionens medlemmar. I Facebookgruppen är det
störst risk att något sådant delas utan tillåtelse, vilket NärU ska ha koll på och rapportera till
gruppadministratören. Godkända annonser skapas på sektionshemsidan och kan sedan delas
till Facebook, i enlighet med kontrakt skrivet med företaget.
\subsubsection{Examensarbeten}
Examensarbeteserbjudanden behöver inte godkännas av NärU eftersom det behövs för
studenternas utbildning. De ska endast publiceras på sektionens hemsida, på en egen sida och
ej på Facebook. Detta på grund av den lilla mängden som detta är intressant för samt för att
begränsa mängden poster i det flödet. Det ska finnas ett formulär på sektionshemsidan där
man kan enkelt mata in information angående ett examensarbete och få det publicerat på
hemsidan, efter godkännande.
\subsection{Studenter söker programmerare}
Då och då erbjuds det extrajobb för programmerare, vid sidan av studierna. Dessa
erbjudanden postas typiskt av privatpersoner i Facebookgruppen men bör inte tillåtas utan
kontroll av NärU. Detta handlar om rekrytering på samma sätt som för många företag, som
betalar pengar för att. Upptäcks dessa poster i Facebookgruppen måste gruppadministratören
meddelas som får ta bort inlägget och kontakta den som postade om vad som gäller och sedan
får NärU sköta kommunikationen.
\subsection{Övrigt}
Vid övrig information som vill spridas till sektionens medlemmar måste kommunikation föras
mellan sektionens ordförande, sekreterare och näringslivsansvarig. Dessa tar sedan beslut om
ett tillvägagångsätt.

\end{document}