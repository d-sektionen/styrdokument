\documentclass{datateknologsektionen-document}

\title{Policy för elektroniskt röstningssystem}
\date{2017-01-20}

\begin{document}
\hspace{0pt}
\vfill
\begin{center}
  \Huge\textbf{Policy för användande av elektroniskt röstningssystem i samband med sektionsmöten}

  \huge Datateknologsektionen

  \large
  Organisationsnummer: 822002-1409

  Version: 1.0

\end{center}
\vfill
\hspace{0pt}
\pagebreak

% \section*{Versionshistorik}
% \begin{footnotesize}
%   \begin{longtabu} to \linewidth { |l|l|X[2]|X| }
%     \hline
%     Version & Datum & Anmärkning & Ansvarig \\ \hline
%     0.1 & 2015-02-01 & Första utkastet & Robin Abrahamsson \\ \hline
%     0.2 & 2015-03-20 & Lagt till versionshistorik, innehållsförteckning och en del förtydliganden & Robin Abrahamsson, Simon Lindén \\ \hline
%     0.3 & 2015-04-23 & Ytterligare förtydligande och omstrukturering & Robin Abrahamsson, Simon Lindén, Jonathan Nyman \\ \hline
%     0.4 & 2017-11-23 & Lade till §2.5 som syftar på den interna köp- och säljgruppen. Flyttade gamla §2.5 till §2.6. & Daniel Roos \\ \hline
%   \end{longtabu}
% \end{footnotesize}

% \pagebreak

\tableofcontents

\pagebreak

\section{Inledning}
Det här är ett dokument som behandlar Datateknologsektionens användande av elektroniska
röstningssystem vid mer formella tillställningar såsom sektionsmöten. Det som står i dokumentet
gäller som kravspecifikation för att ett elektroniskt röstningssystem skall anses godtagbart och
punkterna i avsnitt \S \ref{anvandande} som behandlar användandet av ett sådant bör ses som riktlinjer. Om behov
finns ska detta dokument uppdateras för att spegla rådande förhållanden. Finns det oklarheter skall
studentens intressen sättas i första hand.
\subsection{Bakgrund}
Datateknologsektionens sektionsmöten är ofta utdragna och tidsineffektiva. En väldigt stor tidstjuv
är fastställandet av röstlängden inför röstning. Två rösträknare tillika justeringspersoner ska
oberoende av varandra räkna antal röstberättigade i salen och meddela ordförande om den
enskildes resultat. Vid olika resultat för rösträknarna måste processen göras om från början. En
process som är långt ifrån optimal.
\subsection{Syfte}
Ett väl framtaget elektroniskt röstningssystem syftar till att göra fastställandet av röstlängden en
smidigare process, öka den enskilde medlemmens integritet i röstningsfrågor och göra mötet mer
tillmötesgående för nytillträdda medlemmar i det avseendet att det kan vara svårt att veta hur man
får sin röst hörd på ett formellt sektionsmöte. Syftet med det här dokumentet är att ställa krav på
det elektroniska systemet för att på så sätt undvika att medlemmarnas intressen vad gäller integritet,
tillgänglighet och dylikt inte tillvaratas, nu men även i framtiden.
\section{Formella tekniska krav på det elektroniska röstningssystemet}
\subsection{Identifiera röstberättigade medlemmar}
\label{rostberattigade}
Lösningen skall vara utformad på ett sådant sätt att systemet kan avgöra om en person, medlem eller
icke-medlem, är röstberättigad vid mötet. Som röstberättigad medlem avses person definierad i
stadgarna under \S 2. Systemet skall med hjälp av lämplig identifiering, t.ex. liu-id eller namn kunna
avgöra en sådan fråga.
\subsection{Registrera röstberättigade medlemmar i röstlängden}
\label{registrerarostlangd}
Lösningen skall vara utformad på ett sådant sätt att personer som identifierats som röstberättigad
medlem enligt \S \ref{rostberattigade} skall registreras i röstlängden. Detta kan exempelvis göras genom att personen
temporärt sparas i en lista.
\subsection{Avregistrera röstberättigade medlemmar ur röstlängden}
\label{avregistrerarostlangd}
Lösningen skall vara utformad på ett sådant sätt att personer som har a) identifierats som
röstberättigad medlem enligt \S \ref{rostberattigade} och b) har registrerats enligt \S \ref{registrerarostlangd} också skall kunna avregistreras
ur röstlängden.
\subsection{Fastställande av röstlängden}
\label{faststallarostlangd}
I enlighet med \S \ref{registrerarostlangd} och \S \ref{avregistrerarostlangd} så skall lösningen också kunna tillkännage en skalär på den aktuella
röstlängden.
\subsection{Säkerställande av anonymitet i röstning}
Systemet skall tillämpa gängse krypteringsmetoder och det skall vara omöjligt för någon att varken i
realtid eller i efterhand kunna identifiera enskilda medlemmars erlagda röst i frågor.
\section{Rutiner för användande av systemet}
\label{anvandande}
\subsection{Mötets initiala öppnande}
\label{motetsoppnande}
Om möteslokalen har fler än en ingång så skall det tillses att minst en dörrvakt är vid respektive dörr.
Totalt skall minst en dörrvakt skall utses, förslagsvis från styret, som har till uppgift att kontrollera
samtliga in och utpasserande personer. Det måste alltid sitta en person vid ingången till mötet under
hela mötet. Personer måste alltid registrera sig i röstlängden enligt \S \ref{registrerarostlangd} vid ingång till salen, och
avregistreras vid utgång enligt \S \ref{avregistrerarostlangd}. Personer som inte kan räknas som röstberättigad medlem enligt
\S \ref{rostberattigade} men som ändå önskar deltaga på mötet av någon anledning skall anmäla sig vid presidiet och
personernas närvaro på mötet skall i sedvanlig ordning adjungeras in av mötet. Observera att en
sådan person ej skall föras in i röstlängden ty de inte är röstberättigade.
\subsection{Förtydligande kring tillfälliga deltagare i anslutning till \S \ref{motetsoppnande}.}
Det förekommer ofta att LinTek eller att någon typ av företag önskar hålla en presentation i början
på mötet. Dessa personer är generellt sätt inte röstberättigade medlemmar. Sådana personer
behöver generellt inte adjungeras in eftersom presentationen kan hållas innan mötets formella
öppnande. Om personerna önskar deltaga längre än mötets formella öppnande så måste de såklart
adjungeras in av mötet i sedvanlig ordning.
\subsection{Fastställande av röstlängden}
Systemet skall enligt \S \ref{faststallarostlangd} kunna ange en skalär för röstlängden. Systemet skall anses vara så pass
säkert att det ej bör finnas skäl till misstanke om felaktigheter. Ansvaret ligger dock hos de utsedda
rösträknarna och tillika justeringspersonerna och således, om de önskar, så skall traditionsenlig
röstlängdsräkning göras. I annat fall skall systemets föreslagna röstlängd godkännas av
justeringspersonerna innan det av ordförande tas upp för beslut till mötet.
\subsection{Ajournering av mötet}
Vid matpauser och dylikt så brukar mötet generellt sätt ajourneras. En sådan ajournering innebär att
röstlängden sätts till 0 personer.
\subsection{Återupptagande av mötet}
Efter en ajournering så skall processen beskriven i \S \ref{motetsoppnande} göras om på nytt.
\subsection{Enskilda personer ansluter mitt i mötet (ej i mötets öppnande eller under ajournering)}
I enlighet med \S \ref{motetsoppnande} så är det upp till ”dörrvakten” att kontrollera personens status, dvs om personen
är att betrakta som en röstberättigad medlem eller ej. Om ja så kan dörrvakten lägga till personen i
röstlängden. Om nej så skall denne på ett sådant sätt att det inte stör mötet lämna sina uppgifter till
presidiet och dess eventuella adjungering tas upp av mötet innan något annat beslut kan tas.
\subsection{Enskilda personer lämnar mitt i mötet (ej i ajournering)}
I enlighet med \S \ref{motetsoppnande} så skall dörrvakten kontrollera personen som går ut. Om personen var registrerad
i röstlängden så skall personens namn strykas ur röstlängden i enlighet med \S \ref{avregistrerarostlangd}.

\end{document}