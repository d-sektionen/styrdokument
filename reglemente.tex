\documentclass{datateknologsektionen-document}
\usepackage{amssymb}

\title{Reglemente för Datateknologsektionen}
\date{2021-MM-DD}

\begin{document}

\hspace{0pt}
\vfill

\begin{center}
  \Huge\textbf{Reglemente}

  \huge Datateknologsektionen

  \large
  Organisationsnummer: 822002-1409

  \vspace{4mm}
  (Bilaga till Datateknologsektionens stadgar)

  \vspace{4mm}

  Antagna 2021-MM-DD

\end{center}
\vfill
\hspace{0pt}
\pagebreak

\section*{Versionshistorik}
\begin{footnotesize}
  \begin{longtabu} to \linewidth { |l|X| }
    \hline
    \textbf{Sektionsmöte} & \textbf{Ändring}  \\ \hline
    YYYY-MM-DD & Antagna av sektionsmötet enligt XXXXXX \\ \hline
  \end{longtabu}
\end{footnotesize}

\pagebreak
\tableofcontents

\pagebreak

\section{Policydokument}
Utöver stadgan och detta reglemente kan styrelsen anta och anslå policydokument som
vidare reglerar sektionens verksamhet. Följande policydokument ska finnas:
\begin{itemize}
  \item Datahanteringspolicy
  \item Rekryteringspolicy
\end{itemize}

\section{Sektionsaktiva}
Sektionsaktiv är den som aktivt arbetar i ett eller flera utskott inom sektionen.
När sektionsaktiva ska väljas in ska dessa i första hand vara studerandemedlemmar.
Finns anledning att rekrytera icke-studerandemedlem skall valet motiveras inför
sektionsstyrelsen.

Aktiva medlemmar inom arbetsmarknadsgruppen regleras genom avtal (se \S \ref{arbetsmarknadsgruppen}).

Det åligger samtliga sektionsaktiva att:
\begin{itemize}
  \item Utföra sina uppgifter efter bästa förmåga.
  \item Representera sektionen i en positiv anda utåt.
  \item Göra ett bra överlämnande till sina efterträdare.
  \item Ha riktigt roligt!
\end{itemize}

\section{Sektionsstyrelsen}
Utöver de poster som finns definierade i stadgan finns inom styrelsen följande poster:
\begin{itemize}
  \item Utbildningsutskottets ordförande
  \item Arbetsmiljöombud
  \item 1-3 Ledamöter
\end{itemize}

Styrelsen ansvarar för att ta fram ett måldokument samt anslå detta på hemsidan
innan verksamhetsårets första sektionsmöte.

\subsection{Utbildningsutskottets ordförande}
\label{utbuordf}
Det åligger utbildningsutskottets ordförande att:
\begin{itemize}
  \item Sammankalla utbildningsutskottet.
  \item Hålla sektionsstyrelsen informerad om utbildningsutskottets verksamhet.
  \item Ansvara för att sektionen är representerad i utbildningsrelaterade forum.
  \item Ansvara för och kontrollera utbildningsutskottets arbete.
\end{itemize}

\subsection{Arbetsmiljöombud}
\label{amo}
Det åligger arbetsmiljöombudet att:
\begin{itemize}
  \item Ansvara för att sektionen är representerad i arbetsmiljö- och miljörelaterade forum.
  \item Hålla sektionen informerad om arbetsmiljö- och miljöfrågor.
  \item Samordna miljöansvariga inom sektionen.
\end{itemize}

\subsection{Ledamot}
\label{ledamot}
Det åligger varje ledamot av sektionsstyrelsen att:
\begin{itemize}
  \item Driva projekt fastställda av styrelsen.
\end{itemize}

\section{Kansliet}
Kansliet är ett forum som samlar den operativa verksamheten på sektionen, för
informationsutbyte och samordning. Till kansliet hör förtroendevalda utskottsordföranden
samt sektionsstyrelsen. I händelse av förhinder kan ersättare skickas från utskottet;
företrädesvis en annan förtroendevald.

Kansliet sammanträder minst två gånger per termin enligt kallelse från sektionens
vice ordförande. Sektionens vice ordförande leder sammanträdena och sektionssekreterare
är ständig sekreterare för protokollförda sammanträden.

\section{Förtroendeuppdrag}
Utöver de i stadgan eller ovan angivna förtroendeposterna leds sektionens verksamhet av följande förtroendeposter:
\begin{itemize}
  \item Aktivitetsutskottets ordförande
  \item Aktivitetsutskottets kassör
  \item Arbetsmarknadsgruppens projektledare
  \item Arbetsmarknadsgruppens kassör
  \item Alumniutskottets ordförande
  \item Infoutskottets ordförande
  \item Marknadsföringsutskottets ordförande
  \item Näringslivsutskottets ordförande
  \item Pubutskottets ordförande
  \item Webbutskottets ordförande
  \item Werkmästeriets ordförande
  \item Studienämndsordförande, en för varje utbildningsprogram med nyintag
  \item Festgeneral inför verksamhetsår då sektionen ska högtidlighållas enligt \S \ref{hogtidshallande}.
\end{itemize}
Det åligger sektionsstyrelsen att utse ansvariga personer, internt eller externt, till i denna paragraf angivna poster om de inte tillsatts av sektionsmötet.

\subsection{Tidpunkt för inval}
Ordinarie inval till förtroendeposterna sker vid följande möten.

\subsubsection{Höstmötet}
\begin{itemize}
  \item Arbetsmarknadsgruppens projektledare och kassör
  \item Fadderiets ordförande och kassör
\end{itemize}

\subsubsection{Vintermötet}
\begin{itemize}
  \item Aktivitetsutskottets ordförande och kassör
  \item Damföreningens ordförande och kassör
  \item Festeriets ordförande och kassör
  \item Sektionsstyrelsen
\end{itemize}

\subsubsection{Vårmötet}
\begin{itemize}
  \item Resterande förtroendeposter i reglementet
  \item Revisorer
  \item Valberedning
\end{itemize}

\section{Utskott}
Det åligger ordförande för varje i denna paragraf angivet utskott att leda och organisera arbetet inom utskottet, och
ansvara för utskottet inför sektionsstyrelsen. Det åligger vidare varje förtroendevald
för alla utskott att veta vad som är relevant för dess utskott utifrån det som står i
stadgar samt reglemente. Det åligger även dessa att förmedla denna information till
sina utskottsmedlemmar.

För de utskott som finns definierade i stadgan utses de övriga utskottsmedlemmarna av
styret på förslag från utskottets ordförande. I övrigt utses utskottsmedlemmarna av
utskottets ordförande.

\subsection{Aktivitetsutskottet}
Aktivitetsutskottet ansvarar för att:
\begin{itemize}
  \item Det finns aktiviteter som passar alla oberoende av fysisk kapacitet eller tidigare erfarenheter.
  \item Utskottet ska verka för ökad sammanhållning mellan sektionens medlemmar.
  \item Utskottet ska uppmuntra och stödja medlemmar som vill göra egna aktiviteter.
\end{itemize}

\subsubsection{Projektgrupper}
Aktivitetsutskottet arbetar som en samling projekt som leds av en projektledare.
Projektledare utses av aktivitetsutskottets ordförande, därefter ansvarar enskild
projektledare för eventuell rekrytering till projektgruppen. Projektledaren leder och
organiserar arbetet inom projektet, och ansvarar inför aktivitetsutskottets
ordförande och kassör.

\subsection{Alumniutskottet}
Alumniutskottets uppgift är att på olika sätt främja kontakten mellan sektionen
och de studenter som har avslutat sina studier.

Det åligger alumniutskottet att:
\begin{itemize}
  \item Arbeta för att främja kontakten med sektionens alumnimedlemmar.
  \item Sköta sektionens alumniweb.
\end{itemize}

\subsection{Arbetsmarknadsgruppen}
\label{arbetsmarknadsgruppen}
Arbetsmarknadsgruppen har till uppgift att anordna arbetsmarknadsdagar för studenter
inom program för data och IT.

Verksamheten styrs av det avtal som finns mellan Datateknologsektionen och Linköpings
Y-teknologsektion (Y-sektionen).

Projektledare och kassör för arbetsmarknadsgruppen väljs i samarbete med Y-sektionen.

Projektledaren agerar ordförande.

\subsection{Damföreningen Donna}
Damföreningen är en organisation för sina medlemmar vid Datateknologsektionen i Linköping.
Organisationen skall främja sammanhållningen mellan medlemmarna i damföreningen genom
diverse aktiviteter. Damföreningen ska även aktivt verka för att få fler potentiella
medlemmar att söka sig till de utbildningar som sektionen representerar. Kring
mottagningen ska damföreningen ordna en aktivitet för de nya medlemmarna i damföreningen
så att dessa blir introducerade så snart som möjligt.

\subsubsection{Medlemmar}
Medlem av damföreningen måste anse sig vara av kvinnligt kön eller ickebinär.

\subsection{Ekonomigruppen}
Ekonomigruppen ska arbeta för att stödja och underlätta kassörsarbetet.

Sektionskassören agerar ordförande.

\subsection{Fadderiet Staben}
Fadderiet ska ansvara för sektionens mottagning av de nya studenterna under mottagningsperioden.
Fadderiet ska tillse att varje ny student har en fadder och ett meningsfullt nolleuppdrag
under mottagningsperioden. Alla aktiviteter för de nya studenterna ska bygga på frivillighet.

\subsection{Festeriet D-Group}
Festeriet ska verka för sammanhållning och god stämning inom sektionen. Festeriet ska
anordna ``D-sektionens Öppna Mästerskap i Dart'' (DÖMD), som ska pågå i tre dagar.

\subsubsection{Festerikassör}
Festerikassören ska utöver det som i \S \ref{kassorsataganden} regleras:
\begin{itemize}
  \item Rapportera det ekonomiska läget till sektionsstyrelsen, åtminstone följande gånger:
        \begin{itemize}
          \item En gång på hösten.
          \item En gång på våren, innan DÖMD. Här ska även en översiktlig DÖMD-budget presenteras.
        \end{itemize}
\end{itemize}

\subsection{Festutskottet}
Festutskottet ansvarar för att fira sektionens bildande.

Festgeneralen agerar ordförande.

\subsection{Infoutskottet}
\label{infoutskottet}
Infoutskottets uppgift är att på olika sätt informera sektionsmedlemmar om sektionen.
Hur information ska spridas inom sektionen ska finnas specificerat i sektionens
policy för informationsspridning.

Infoutskottet ansvarar även för att:
\begin{itemize}
  \item Bistå sektionen med möjligheten att dokumentera dess aktiviteter.
  \item Information sprids till medlemmarna via regelbundna infomail.
  \item Sköta sektionens grafiska profil och säkerställa att den efterlevs.
  \item Bistå med grafiskt stöd.
  \item Producera en fotokatalog som läggs upp på sektionens hemsida.
  \item Producera sektionens sidor i LiTHanian.
\end{itemize}

\subsection{Marknadsföringsutskottet}
Marknadsföring mot gymnasieskolor ska bedrivas för alla de utbildningsprogram sektionen representerar.

Marknadsföringsutskottet ska ha samarbete med åtminstone Avdelningen för externa relationer,
Tekniska fakultetskansliet vid Linköpings universitet och Programnämnden för data- och medieteknik.

Marknadsföringsutskottet ansvarar även för att:
\begin{itemize}
  \item Anordna inspirationsdagar för gymnasieelever.
  \item Sprida information om hemmissionering till sektionens medlemmar.
  \item Rekrytera sektionsmedlemmar till rekryteringsevent.
  \item Uppdatera sektionens och Programnämnden för data- och medietekniks gymnasieinformation på webben.
  \item Representera sektionen i marknadsföringsrelaterade forum.
\end{itemize}

\subsection{Näringslivsutskottet}
Näringslivsutskottet ansvarar för att:
\begin{itemize}
  \item Bedriva marknadsföring av sektionen mot företag.
  \item Få in sponsorpengar på minst det belopp som anges i budgeten.
  \item Arkivera sponsoravtal och se till att de följs.
\end{itemize}

\subsection{Pubutskottet}
Utskottet ska anordna pubar för sektionens medlemmar.

\subsection{Utbildningsutskottet}
Utbildningsutskottet driver alla sektionens utbildningsfrågor och ansvarar för
utbildningsbevakningen för alla de utbildningsprogram sektionen representerar.
Utbildningsutskottet består minimalt av utbildningsutskottets ordförande och
sektionens förtroendevalda studienämndsordförande.

Det åligger utbildningsutskottet att:
\begin{itemize}
  \item Informera programmens studenter om utbildningsfrågor.
  \item Lyfta studenters frågor i utbildningsärenden.
  \item Aktivt arbeta för att förbättra sitt eget och framtida utbildningsutskotts arbete.
  \item Ta fram sektionens nominering till LinTeks pedagogiska pris ``Gyllene Moroten'' i enlighet med rådande avtal med LinTek.
\end{itemize}

\subsubsection{Studienämndsordförande}
Utbildningsbevakningen ska ledas av studienämndsordförande för varje utbildningsprogram
sektionen representerar som gör nyintagning. Varje studienämndsordförande bör bedriva
studier vid det program som denne ansvarar för. Bevakningen sköts minimalt genom utvärderingsmöten

Studienämndsordföranden representerar programmens studenter i följande sammanhang:
\begin{itemize}
  \item Arbetet som utförs i Programnämnden för data-och medieteknik (DM) vid Linköpings Tekniska Högskola vid Linköpings universitet.
  \item Varje studienämndsordförande deltar i det arbete som utförs i Programplaneringsgruppen för det program hen ansvarar för.
  \item Studienämndsordföranden sitter i LinTeks Utbildningsråd (LinTek UR).
\end{itemize}

Studienämndsordförande ska hålla samtliga studenter vid respektive utbildningsprogram
informerade i studiebevakningsfrågor samt vara lyhörda för den vanlige studentens
förslag och idéer samt framföra dessa till berörd programplaneringsgrupp och DM-nämnden.

\subsection{Webbutskottet}
Utskottet ska underhålla och vidareutveckla sektionens webbsidor.

Utskottet ansvarar även för att:
\begin{itemize}
  \item Handha och ansvara för sektionens medlemsregister.
\end{itemize}

\subsection{Werkmästeriet}
Det åligger werkmästeriet att:
\begin{itemize}
  \item Handha sektionens egendomar och tillse att nödvändigt reparations- och underhållsarbete utförs.
  \item Ansvara för sektionslokalerna.
  \item Handha sektionens medaljer.
  \item Ansvara för sektionens märken i Märkesbacken.
  \item Ansvara för sektionens väggmålningar på C-huset.
  \item Ansvara för underhåll av sektionens maskotar på Corson.
\end{itemize}

\section{Ekonomisk verksamhet}
\subsection{Budget}
Datateknologsektionen är en budgetenhet vilken löper över verksamhetsåret och
inkluderar all sektionens verksamhet.

Sektionsmötet fastslår en rambudget där respektive resultatenhets inkomster och
utgifter redovisas. Attesträtt för resultatenheterna beslutas av sektionsstyrelsen.
Det ska vid presentation av budget och budgetuppföljning vara möjligt att göra
jämförelse med föregående års budget och utfall. Rapportering av budgetuppföljning
ska ske till sektionsmötet.

Om resultatet för en resultatenhet beräknas avvika negativt med mer än 50 \% eller
100 000 kronor, måste detta anmälas till sektionsmötet. Om likaledes resultatet för
budgetenheten som helhet beräknas avvika negativt med mer än 50 \% eller 150 000 kronor,
måste detta anmälas till sektionsmötet.

Vid eventuellt överskott skall detta användas i följande ordning:
\begin{itemize}
  \item Säkerhetsställa likviditet för nästa verksamhetsår.
  \item Investeringar under nästa verksamhetsår.
  \item Fonderas till någon av sektionens existerande fonder eller till ny fond.
\end{itemize}

Resultatenheten arbetsmarknadsgruppen hanteras enligt avtal (se \S \ref{arbetsmarknadsgruppen}.)

\subsection{Bokföring}
All fysisk bokföring skall förvaras i sektionens lokaler med undantag för kortare
tidsperioder om bokföringsarbetet eller revision så kräver.

Syftena med bokföringen, utan prioritetsordning, är följande:
\begin{itemize}
  \item Att möjliggöra kontroll av sektionens verksamhet, genom revisorernas försorg.
  \item Att underlätta arbetet med att uppnå budget.
  \item Att underlätta för nästkommande års verksamhet.
\end{itemize}

\subsection{Kassörsåtaganden}
\label{kassorsataganden}
Det åligger alla förtroendevalda kassörer i sektionen inom ramen för sin roll att:
\begin{itemize}
  \item Sköta den ekonomiska verksamheten enligt denna huvudparagraf
  \item Föra kund- och leverantörsreskontra, betala fakturor i tid samt följa upp icke betalda fordringar.
  \item Att på vårmötet under sitt verksamhetsår presentera budget för kommande verksamhetsår.
  \item Att på höstmötet efter sitt avslutade verksamhetsår presentera ekonomisk sammanställning.
\end{itemize}

\section{Sektionens resurser}
\subsection{Sektionens rum}
Sektionen har två rum. Ett rum ska fungera som kontor och ett ska fungera som samlingsrum.
Sektionens rum ska skötas enligt de lokalsponsoravtal som eventuellt finns.

\subsubsection{Kontoret}
Sektionens kontor är till för sektionsaktivas arbete.

\subsubsection{Samlingsrummet}
Rummet bör hållas öppet på vardagar mellan kl. 8 och kl. 17 under läsperioder.
Werkmästeriet ansvarar för att rummet hålls öppet för sektionens medlemmar.

\subsection{Förråd}
Sektionen har förråd som används för materialförvaring och dokumentförvaring.

\subsection{Sektionens bil}
Sektionen har en bil som används till sektionsarbete och hyrs ut till sektionens medlemmar samt
utomstående. Bilen hyrs ut via bokningssystemet på sektionens webbplats.

\subsection{Övriga resurser}
Samtliga tillgångar inom organisationen tillhör sektionen som helhet och kan disponeras
av samtliga utskott i samråd med förvaltande utskott. Undantaget är arbetsmarknadsgruppen
vars tillgångar utöver detta regleras genom avtal (se \S \ref{arbetsmarknadsgruppen}).

\section{Sektionens fonder}
\label{fonder}
Sektionens fonder ska förvaltas på det sätt som anges i respektive nedanstående paragrafer, eller
enligt de regler för respektive fond som fastlagts vid fondens instiftande. Om det skett
transaktioner inom fonden under året ska ett bokslut för fonden bifogas den förvaltandes
årsredovisning.

\subsection{Bilfond}
\subsubsection{Avsättning}
Till sektionens bilfond får medel avsättas från verksamheter som gynnas av att sektionen har en
fungerande bil.

\subsubsection{Användning}
Fondens medel ska användas för att säkerhetsställa att sektionen har en fungerande bil.
Begäran om användning av fondmedel görs av werkmästeriets ordförande till beslutsfattande
organ. Det åligger sektionsstyrelsen att aktivt informera werkmästeriets ordförande om
fondens existens och möjligheter. Begäran om användning kan göras om det i budget inte
ryms att säkerhetsställa bilens funktion.

\subsubsection{Beslut om användande}
Beslutsfattande organ är sektionsstyrelsen.

\subsection{Jubileumsfond}
\subsubsection{Avsättning}
Till sektionens jubileumsfond får medel avsättas som härrör från sponsring, bidrag
eller gåvor erhållna av företag, institutioner eller privatpersoner.

\subsubsection{Användning}
Fondens medel ska användas för att fira sektionens bildande. Användningen ska föregås
av en framlagd och godkänd budget från festgeneralen till beslutsfattande organ. Det
åligger sektionsstyrelsen att aktivt informera festgeneralen om fondens existens och
möjligheter. Noteras att extern finansiering får användas i samband med fondens användande.

\subsubsection{Beslut om användande}
Beslutsfattande organ är sektionsstyrelsen.

\section{Insignia}
\subsection{Emblem}
Sektionen representeras med ett officiellt emblem framröstat av sektionsmötet.

\subsection{Grafisk profil}
Sektionens grafiska profil reglerar sektionens visuella uttryck. Den grafiska
profilen förvaltas av infoutskottet (se \S \ref{infoutskottet}) och finns tillgänglig på sektionens hemsida.

\subsection{Sektionsband}
Sektionen har ett sektionsband som bärs till högtidsdräkt, smoking eller mörk kostym
under högtidstillfällen av den som är medlem i sektionen. Sektionsbandet kan även
bäras som rosett. Bandet är 36~mm brett och utformat med sektionens färger enligt
färgfördelning 10~mm brunt / 16~mm gult / 10~mm brunt.

\subsection{Medaljer}
Medaljer delas ut årligen till personer som engagerat sig för sektionen som ett
bevis på uppskattning efter verksamhetsårets slut. Valörer högre än förtjänst
utdelas efter beviljad ansvarsfrihet. Medaljer bärs företrädesvis i band med samma
utformning som sektionsbandet.

\subsubsection{Valörer}
För att det inte ska skilja från år till år vilken medalj en viss befattning resulterar
i ska följande valörer vara normgivande.

\begin{itemize}
  \item Sektionens guldmedalj är sektionens högsta utmärkelse och utdelas av sektionsmötet
        till den som utses till hedersmedlem eller till en person som under lång tid och på
        ett beundransvärt sätt arbetat för sektionen i dess mål och strävanden.

  \item Sektionens silvermedalj är sektionens högsta utmärkelse för engagemang inom sektionen
        och utdelas av sektionsmötet för sitt andra fullgjorda förtroendeuppdrag.

  \item Sektionens bronsmedalj utdelas av sektionsmötet för sitt första fullgjora förtroendeuppdrag.

  \item Sektionens förtjänstmedalj utdelas av sektionsmötet för att ha varit aktiv under ett verksamhetsår.
        Förtjänstmedalj kan också utdelas efter särskilt motiverat beslut till sektionsstyrelsen.
\end{itemize}

\section{Högtidlighållanden}
\label{hogtidshallande}
Datateknologsektionen bildades 27 januari 1976, d.v.s. året efter att D-programmet bildades.

Utbildningsprogramen bildades följande år:
\begin{itemize}
  \item Civilingenjörsprogrammet i Datateknik 1975
  \item Datavetenskapsprogrammet 1982
  \item Civilingenjörsprogrammet i Informationsteknologi 1995
  \item Kandidatprogrammet i Innovativ Programmering 2007
  \item Civilingenjörsprogrammet i Mjukvaruteknik 2013
\end{itemize}

Datateknologsektionens bildande bör firas minst var 5:e år. Lämpliga årtal för jubileum
ges av \(1976 + 5k\) eller \(1976 + 2^k, k \in \mathbb{N}\).

Utbildningsprogrammens bildande bör uppmärksammas minst var 5:e år, vilket sektionsstyrelsen ansvarar för.


\end{document}