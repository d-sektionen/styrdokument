\documentclass{datateknologsektionen-document}
\usepackage{amssymb}

\title{Reglemente för Datateknologsektionen}
\date{2019-04-15}

\begin{document}

\hspace{0pt}
\vfill
\begin{center}

\Huge{
  \textbf{Reglemente} för
  
  Datateknologsektionen
  
  Läsåret 2019-2020
}

  \vspace{30pt}

\Large{(Bilaga till Datateknologsektionens stadgar)}
  
\end{center}
\vfill
\hspace{0pt}

\pagebreak

\tableofcontents

\pagebreak

\section{Samtliga sektionsaktiva}
Det åligger samtliga sektionsaktiva att:
\begin{itemize}
  \item Utföra sina uppgifter efter bästa förmåga.
  \item Representera sektionen i en positiv anda utåt.
  \item Göra ett bra överlämnande till sina efterträdare.
  \item Ha riktigt roligt!
\end{itemize}
\section{Sektionsstyrelsen}
Det åligger sektionsstyrelsen att:
\begin{itemize}
  \item Söka kandidater till Valberedningens ordförande.
\end{itemize}
Sektionsstyrelsen består av följande medlemmar:
\begin{itemize}
  \item Ordförande (Se \S \ref{ordforande})
  \item Vice ordförande (Se \S \ref{viceordforande})
  \item Kassör (Se \S \ref{styrelsekassor})
  \item Sekreterare (Se \S \ref{sekreterare})
  \item Utbildningsutskottets ordförande (Se \S \ref{utbuordf})
  \item Arbetsmiljöombud (Se \S \ref{amo})
  \item 1-3 Ledamöter (Se \S \ref{ledamot})
\end{itemize}
\subsection{Ordförande}
\label{ordforande}
Det åligger sektionens ordförande att:
\begin{itemize}
  \item Leda sektionsstyrelsens arbete.
  \item Leda sektionens sammanträden (sektionsmöten).
  \item Representera sektionen i sammanträden anordnade av LinTek kopplade till ordförandeposten.
  \item Se till att sektionsstyrelsens och utskottens arbete sker i enlighet med stadgarna.
  \item Ta fram ett måldokument samt anslå detta på hemsidan innan verksamhetsårets första sektionsmöte.
\end{itemize}
\subsection{Vice ordförande}
\label{viceordforande}
Det åligger vice ordförande att:
\begin{itemize}
  \item Avlasta ordföranden.
  \item Fungera som ställföreträdare för ordföranden vid dennes frånvaro.
  \item Verka för att stärka sammanhållningen bland de sektionsaktiva.
  \item Sammankalla alla förtroendevalda poster minst två gånger per termin för att samordna sektionens arbete.
  \item Ansvara för och kontrollera följande poster:
  \begin{itemize}
    \item Valberedningens ordförande. (Se \S \ref{valordforande}).
    \item Art director. (Se \S \ref{ad}).
  \end{itemize}
\end{itemize}

\subsection{Kassör}
\label{styrelsekassor}
Se \S \ref{sektionskassor}.
\subsection{Sekreterare}
\label{sekreterare}
Det åligger sekreteraren att:
\begin{itemize}
  \item Skriva protokoll för varje möte och vidarebefordra dem enligt stadgarna. (Se \S 5.10 och \S 7.7 i stadgarna).
  \item Säkerställa att information från sektionsstyrelsen kommer ut till sektionen.
\end{itemize}
\subsection{Ledamot}
\label{ledamot}
Det åligger varje ledamot av sektionsstyrelsen att:
\begin{itemize}
  \item Vara sektionsstyrelsen behjälplig
\end{itemize}
\subsection{Utbildningsutskottets ordförande}
\label{utbuordf}
Det åligger utbildningsutskottets ordförande att:
\begin{itemize}
  \item Sammankalla utbildningsutskottet (Se \S \ref{utbu}).
  \item Protokollföra utbildningsutskottet sammanträden och vidarebefordra viktig information till sektionsstyrelsen.
  \item Vara sektionsstyrelsen och utbildningsutskottets ledamöter behjälplig.
  \item Ansvara för och kontrollera utbildningsutskottets arbete.
\end{itemize}
\subsection{Arbetsmiljöombud}
\label{amo}
Det åligger arbetsmiljöombudet att:
\begin{itemize}
  \item Anta sig uppdrag för att främja sektionens medlemmar inom arbetsmiljöfrågor.
  \item Föra sektionens talan och driva sektionens frågor på AMO-möten.
  \item Informera sektionsstyrelsen om vad som beslutas på AMO-möten.
  \item Föra sektionens talan och driva sektionens frågor i studentsamverkansgruppen.
  \item Informera sektionens medlemmar om arbetsmiljöfrågor.
  \item Föra sektionens talan och driva sektionens frågor på MO-möten.
  \item Informera sektionsstyrelsen om vad som beslutas på MO-möten.
  \item Informera sektionens medlemmar om miljöfrågor.
\end{itemize}
\subsection{Adjungeringar}
Ständigt adjungerade med närvaro-, yttrande- och förslagsrätt är:
\begin{itemize}
  \item Samtliga innevarande och nästkommande förtroendevalda.
  \item Revisorer.
\end{itemize}
\section{Kassörer}
\subsection{Bokföring}
All bokföring skall förvaras i sektionens lokaler med undantag för kortare tidsperioder om bokföringsarbetet eller revision så kräver.
\subsubsection{Syfte med bokföring}
\label{bokforingssyfte}
Syftena med bokföringen, utan prioritetsordning, är följande:
\begin{itemize}
  \item Att möjliggöra kontroll av sektionens verksamhet, genom revisorernas försorg.
  \item Att underlätta arbetet med att uppnå budget.
  \item Att underlätta för nästkommande års verksamhet.
\end{itemize}
\subsection{Sektionskassör}
\label{sektionskassor}
Det åligger sektionskassören att förutom det som står i stadgarna:
\begin{itemize}
  \item Sköta sektionens bokföring i enlighet med \S \ref{bokforingssyfte}.
  \item Föra kund- och leverantörsreskontra, betala fakturor i tid samt följa upp icke betalda fordringar.
  \item Inventarie/lagerföra sektionens tillgångar, med undantag för festeriets och fadderiets.
  \item Ha en övergripande bild av sektionens och alla dess utskotts ekonomi.
  \item Sammankalla alla kassörer minst en gång per termin för att samordna sektionens ekonomiska arbete.
  \item Se till att reglerna för sektionens fonder följs.
  \item Vid varje läsperiod eller vid anmodan redovisa sektionens ekonomiska ställning.
  \item Ansvara för utkvittering av sektionens nycklar.
\end{itemize}
\subsection{Festerikassör}
\label{festerikassor}
Det åligger festerikassören att:
\begin{itemize}
  \item Sköta festeriets bokföring i enlighet med \S \ref{bokforingssyfte}.
  \item Budgetera för och följa upp resultat för D-Groups arrangemang. Budgetarbetet för de enskilda arrangemangen, med undantag för DÖMD, kan dock utföras av annan utskottsmedlem, under överinseende av festerikassören.
  \item Föra kund- och leverantörsreskontra, betala fakturor i tid samt följa upp icke betalda fordringar.
  \item Inventarie/lagerföra festeriets tillgångar.
  \item Informera sektionsstyrelsen om festeriets ekonomiska läge, åtminstone följande gånger:
  \begin{itemize}
    \item En gång på hösten.
    \item En gång på våren, innan DÖMD. Här ska även en översiktlig DÖMD-budget presenteras.
  \end{itemize}
  \item På vårmötet presentera ett budgetförslag för nästkommande verksamhetsår.
  \item På höstmötet året efter sitt verksamhetsår presentera:
  \begin{itemize}
    \item Ekonomisk berättelse för föregående års festeriverksamhet. I denna ska även ingå en kort sammanfattning av hur övriga plikter utförts, till exempel överlämnandet till nästa års festerikassör.
  \end{itemize}
\end{itemize}

\subsection{Fadderikassör}
\label{fadderikassor}
Det åligger fadderikassören att:
\begin{itemize}
  \item Sköta fadderiets bokföring i enlighet med \S \ref{bokforingssyfte}.
  \item Föra kund- och leverantörsreskontra, betala fakturor i tid samt följa upp icke betalda fordringar.
  \item Rapportera det ekonomiska läget till sektionsstyrelsen.
  \item På vårmötet presentera en budget för nästkommande verksamhetsår.
  \item Inventarie/lagerföra fadderiets tillgångar.
  \item På höstmötet året efter sitt verksamhetsår presentera:
  \begin{itemize}
    \item Ekonomisk berättelse för föregående års fadderiverksamhet. I denna ska även ingå
    en kort sammanfattning av hur övriga plikter utförts, till exempel överlämnandet till
    nästa års fadderikassör.
  \end{itemize}
\end{itemize}

\subsection{Kassör för Arbetsmarknadsgruppen}
\label{amgkassor}
Det åligger kassören för Arbetsmarknadsgruppen att:
\begin{itemize}
  \item Sköta arbetsmarknadsgruppens bokföring i enlighet med \S \ref{bokforingssyfte}.
  \item Föra kund- och leverantörsreskontra, betala fakturor i tid samt följa upp icke betalda fordringar.
  \item Rapportera det ekonomiska läget till sektionsstyrelsen, åtminstone en gång innan arbetsmarknadsdagarna.
  \item På vårmötet varje udda år presentera ett budgetförslag för nästkommande verksamhetsår.
\end{itemize}

\subsection{Damföreningens kassör}
\label{damforeningskassor}
Det åligger Damföreningens kassör att:
\begin{itemize}
  \item Sköta utskottets bokföring i enlighet med \S \ref{bokforingssyfte}.
  \item På vårmötet presentera ett budgetförslag för nästkommande verksamhetsår.
  \item Betala fakturor i tid samt följa upp icke betalda fordringar.
  \item Rapportera det ekonomiska läget till sektionsstyrelsen.
\end{itemize}

\subsection{Aktivitetskassör}
\label{aktukassor}
Det åligger aktivitetskassören att:
\begin{itemize}
  \item Sköta aktivitetsutskottets bokföring i enlighet med \S \ref{bokforingssyfte}.
  \item På vårmötet presentera ett budgetförslag för nästkommande verksamhetsår.
  \item Betala fakturor i tid samt följa upp icke betalda fordringar.
  \item Rapportera det ekonomiska läget till sektionsstyrelsen.
\end{itemize}
\section{Revisorer}
Förutom det som finns specificerat i stadgarna åligger det revisorerna att:
\begin{itemize}
  \item Kontinuerligt under verksamhetsåret revidera sektionen och dess utskott.
  \item Informera sektionsstyrelsen om resultatet av denna revision.
  \item Innan och under verksamhetsåret informera de förtroendevalda hur revisionen kommer att genomföras och vilka krav som ställs på bokföring och verksamhet.
\end{itemize}
Styrelsemötet beslutar om utbetalning av revisorsarvode efter att revisorerna presenterat
revisionsberättelse för sektionen och dess utskott.

\section{Övriga förtroendeuppdrag}

Det åligger sektionsstyrelsen att utse ansvariga personer, internt eller externt, till följande poster
om de inte tillsatts under sektionsmötet:
\subsection{Festerichef}
Det åligger festerichefen att:
\begin{itemize}
  \item Vara ordförande för festeriet. (Se \S \ref{festeriet}).
\end{itemize}
\subsection{Fadderigeneral}
Det åligger fadderigeneralen att:
\begin{itemize}
  \item Vara ordförande för fadderiet. (Se \S \ref{fadderiet}).
\end{itemize}
\subsection{Damföreningens ordförande}
Det åligger damföreningens ordförande att:
\begin{itemize}
  \item Vara ordförande för damföreningen. (Se \S \ref{damforeningen}).
\end{itemize}
\subsection{Marknadsföringsutskottets ordförande}
Det åligger marknadsföringsutskottets ordförande att:
\begin{itemize}
  \item Vara ordförande i marknadsföringsutskottet. (Se \S \ref{mafu}).
\end{itemize}
\subsection{Näringslivsansvarig}
Det åligger näringslivsansvarig att:
\begin{itemize}
  \item Vara ordförande i näringslivsutskottet. (Se \S \ref{naru}).
\end{itemize}
\subsection{Aktivitetshanterare}
Det åligger aktivitetshanteraren att:
\begin{itemize}
  \item Vara ordförande i aktivitetsutskottet. (Se \S \ref{aktu}).
\end{itemize}

\subsection{Alumniutskottets ordförande}
Det åligger alumniutskottets ordförande att:
\begin{itemize}
  \item Vara ordförande i alumniutskottet. (Se \S \ref{alumni}).
\end{itemize}
\subsection{Projektledare för arbetsmarknadsgruppen}
Det åligger projektledaren för arbetsmarknadsgruppen att:
\begin{itemize}
  \item Leda projektgruppen för arbetsmarknadsgruppen. (Se \S \ref{amg}).
\end{itemize}
\subsection{Pubutskottets ordförande}
Det åligger pubutskottets ordförande att:
\begin{itemize}
  \item Vara ordförande i pubutskottet. (Se \S \ref{pubu}).
\end{itemize}

\subsection{Valberedningens ordförande}
\label{valordforande}
Det åligger valberedningens ordförande att:
\begin{itemize}
  \item Vara ordförande för valberedningen. (Se \S \ref{val}).
\end{itemize}
\subsection{Festgeneral}
Det åligger festgeneralen att:
\begin{itemize}
  \item Genomföra ett högtidlighållande av sektionens bildande. (Se \S \ref{hogtid}).
\end{itemize}
\subsection{Art director}
\label{ad}
Sektionens Art Director konsulterar sektionsaktivas tryckansvariga för att underlätta
tryckansvarigas arbete. Art Directorn jobbar också för att sektionens samtliga trycksaker ska få ett
enhetligt utseende.
\subsection{Studienämndsordförande}
Varje utbildningsprogram sektionen representerar som gör nyintagning ska ha en
studienämndsordförande (Se \S \ref{snordf}).

\subsection{Intendent}
\label{intendent}
Det åligger intendenten att:
\begin{itemize}
  \item Vara ordförande i werkmästeriet. (Se \S \ref{werk}).
\end{itemize}

\subsection{Webb-/infochef}
Det åligger Webb-/infochefen att:
\begin{itemize}
  \item Vara ordförande i webb-/infoutskottet. (Se \S \ref{webbinfo}).
\end{itemize}


\section{Sektionens fonder}
Sektionens fonder ska förvaltas på det sätt som anges i respektive nedanstående reglemente, eller
enligt de regler för respektive fond som fastlagts vid fondens instiftande. Om det skett
transaktioner inom fonden under året ska ett bokslut för fonden bifogas den förvaltandes
årsredovisning.
\subsection{Festeriets katastroffond}
\subsubsection{Syfte}
Syftet med festeriets katastroffond är att festeriet ska slippa budgetera för icke förutsägbara och
icke påverkbara kostnader i samband med arrangemang.
\subsubsection{Avsättning}
Till fonden får vinstmedel avsättas som härrör från festeriets verksamhet. Fonden får maximalt
uppgå till ett prisbasbelopp. Sektionsstyrelsen beslutar om avsättning till fonden.
Då användning av fondens medel skett bör ett så snabbt återställande av fondens belopp som
möjligt eftersträvas.
\subsubsection{Användning}
Fondens medel får användas för att täcka icke förutsägbara och icke påverkbara kostnader i
festeriets verksamhet. Begäran om användning av fondmedel görs av Festeriets ordförande eller
kassör till sektionsstyrelsen, som beslutar om användning. Begäran om
användning kan göras efter det att kostnaden konstaterats.

\subsubsection{Redovisning}
Ett årsbokslut för fonden ska finnas som bilaga till festeriets bokslut samt presenteras
tillsammans med detta på höstmötet nästkommande verksamhetsår.
\subsection{Sponsorpoolen}
\subsubsection{Avsättning}
Till sponsorpoolen får medel avsättas som härrör från sponsring, bidrag eller gåvor erhållna av
företag, institutioner eller privatpersoner.
\subsubsection{Användning}
Villkor för användning av fondens medel är att de ska komma sektionens
medlemmar tillgodo. Användning ska föregås av ett skriftligt äskande ställt till beslutsfattande
organ där anledningen till äskandet beskrivs och medlens användning specificeras. Det åligger
sektionsstyrelsen att aktivt informera sektionens medlemmar om
fondens existens och möjligheter. Noteras att för ärenden där möjlighet finns till alternativ
extern finansiering bör sådan givetvis användas istället.
\subsubsection{Beslut om användning}
Beslutsfattande organ är sektionsstyrelsen.
\subsection{Alumnifonden}
\subsubsection{Avsättning}
Till alumnifonden får medel avsättas som härrör från sponsring, bidrag eller gåvor erhållna av
företag, institutioner eller privatpersoner.
\subsubsection{Användning}
Villkor för användning av fondens medel är att de ska komma sektionens
medlemmar tillgodo. Användning ska föregås av ett skriftligt yrkande ställt till beslutsfattande
organ där anledningen till yrkandet beskrivs. Det åligger sektionsstyrelsen att
aktivt informera sektionens medlemmar om fondens existens och möjligheter.
Noteras att för ärenden där möjlighet finns till alternativ extern finansiering bör sådan givetvis
användas istället. Det åligger beslutsfattande organ att presentera medlens användning på
sektionens hemsida.
\subsubsection{Beslut om användning}
Beslutsfattande organ är sektionsstyrelsen.


\subsection{Fadderiets riskfond}
\subsubsection{Avsättning}
Till fadderiets riskfond får vinstmedel avsättas som härrör från verksamhet bedriven av
sektionens fadderi. Fonden får uppgå till ett belopp av maximalt ett kvarts
prisbasbelopp. Efter användning ska fondens belopp snarast restitueras till ett kvarts
prisbasbelopp.
\subsubsection{Användning}
Fondens medel får användas för att täcka icke förutsägbara och icke påverkbara kostnader i
fadderiets verksamhet. Begäran om användning av fondmedel görs av fadderiets general eller
kassör till sektionsstyrelsen. Fondens syfte är att fadderiet inför sina egna
arrangemang inte ska behöva budgetera oförutsägbara kostnader. Begäran om användning kan
göras efter det att kostnaden konstaterats, med fördel i slutet av verksamhetsåret.
\subsubsection{Beslut om användning}
Beslutsfattande organ är sektionsstyrelsen.
\subsection{Sektionens Bilfond}
\subsubsection{Avsättning}
Till sektionens bilfond får vinst avsättas från verksamheter som gynnas av att sektionen har en
fungerande bil.
\subsubsection{Användning}
Bilfonden skall användas för att kunna införskaffa en ny bil till sektionen, den kan även
användas för att reparera en befintlig bil om detta anses lämpligt.
\subsubsection{Beslut om användande}
Beslutsfattande organ är sektionsstyrelsen.

\section{Sektionens resurser}
\subsection{Sektionens rum}
Sektionen har två rum. Ett rum ska fungera som kontor och ett ska fungera som samlingsrum.
Sektionens rum ska skötas enligt de lokalsponsoravtal som finns.
\subsubsection{Kontoret}
Sektionens kontor är till för sektionsaktivas arbete.
\subsubsection{Samlingsrummet}
Rummet bör hållas öppet på vardagar mellan kl. 8 och kl. 17 under läsperioder.
Werkmästeriet ansvarar för att rummet hålls öppet för sektionens medlemmar.
\subsubsection{Förråd}
De förrådsrum sektionen har tillgång till ska hållas städade och i ordnat skick.
\subsection{Sektionens bil}
Sektionen har en bil som används till sektionsarbete och hyrs ut till sektionens medlemmar samt
utomstående. Sektionens bilansvarig ansvarar för drift samt underhåll av bilen. Bilen hyrs ut via
bokningssystemet på sektionens webbplats.

\section{Information}
För information inom sektionen ansvarar Webb-/infoutskottet (Se \S \ref{webbinfo}).

Hur information ska spridas inom sektionen ska finnas specificerat i sektionens policy för
informationsspridning.
\subsection{Sektionens policy för informationsspridning}
Sektionen har en policy för informationsspridning som finns på sektionens hemsida. Syftet med
dokumentet är att underlätta beslut av hur information ska spridas.
\section{Studiebevakning}
\subsection{Uppdrag}
Studiebevakning ska bedrivas för alla de utbildningsprogram sektionen representerar.
\subsection{Organisation}
\subsubsection{Utbildningsutskottet}
\label{utbu}
Utbildningsutskottet driver alla sektionens utbildningsfrågor och ansvarar för
utbildningsbevakningen. Utbildningsutskottet består minimalt av utbildningsutskottets
ordförande och sektionens förtroendevalda studienämndsordförande.
\paragraph{Utbildningsutskottets uppgifter}
Det åligger utbildningsutskottet att:
\begin{itemize}
  \item Informera programmens studenter om utbildningsfrågor
  \item Lyfta studenters frågor i utbildningsärenden
  \item Aktivt arbeta för att förbättra sitt eget och framtida utbildningsutskotts arbete
  \item Ta fram sektionens nominering till LinTeks pedagogiska pris ”Gyllene Moroten” i enlighet med eventuellt avtal med LinTek
\end{itemize}
\subsubsection{Studienämndsordförande}
\label{snordf}
Utbildningsbevakningen ska ledas av studienämndsordförande för varje utbildningsprogram
sektionen representerar som gör nyintagning. Varje studienämndsordförande bör bedriva
studier vid det program som denne ansvarar för. Bevakningen sköts minimalt genom
utvärderingsmöten (Se \S \ref{utvarderingsmoten}).

Studienämndsordföranden representerar programmens studenter i följande sammanhang:
\begin{itemize}
  \item Arbetet som utförs i Programnämnden för data-och medieteknik (DM) vid Linköpings Tekniska Högskola vid Linköpings universitet.
  \item Varje studienämndsordförande deltar i det arbete som utförs i Programplaneringsgruppen för det program hen ansvarar för.
  \item Studienämndsordföranden sitter i LinTeks Utbildningsråd (LinTek UR)
\end{itemize}
\subsection{Utvärderingsmöten}
\label{utvarderingsmoten}
För att uppfylla det avtal som sektionen skriver med LinTek angående studiebevakning ska
samtliga studienämndsordförande, minst en gång per läsperiod, se till att alla obligatoriska kurser
utvärderas. Efter bästa förmåga ska dessutom valbara kurser utvärderas. Utvärderingarna ska
skötas av klassrepresentanterna. Under utvärderingsmötena ska protokoll föras för att sedan
distribueras till berörda parter.
\subsection{Information}
Studienämndsordförande ska hålla samtliga studenter vid respektive utbildningsprogram
informerade i studiebevakningsfrågor samt vara lyhörda för den vanlige studentens förslag och
idéer samt framföra dessa till berörd programplaneringsgrupp och DM-nämnden.
\section{Utskott}
Det åligger varje ordföranden och kassör för alla utskott att veta vad som är relevant för dess
utskott utifrån det som står i stadgar samt reglemente. Det åligger även dessa att förmedla denna
information till sina utskottsmedlemmar.
\subsection{Festeriet}
\label{festeriet}
\subsubsection{Uppdrag}
Festeriet ska verka för sammanhållning och god stämning inom sektionen. Festeriet ska ha som
ekonomisk målsättning att efter avslutat verksamhetsår uppvisa ett nollresultat. Festeriet ska
anordna ``D-sektionens Öppna Mästerskap i Dart'' (DÖMD), som ska pågå i tre dagar.
\subsubsection{Organisation}
\paragraph{Festerichef}
Festerichefen leder och organiserar festeriets arbete.
\paragraph{Festerikassör}
Se \S \ref{festerikassor}.
\paragraph{Utskottet}
Festeriet skall i första hand bestå av sektionsmedlemmar. Finns anledning att rekrytera icke-
sektionsmedlem skall valet motiveras inför sektionsstyrelsen. Utskottsmedlemmar bortsett
från festerikassör utses av sektionsstyrelsen på förslag av festerichefen.
\subsection{Fadderiet}
\label{fadderiet}
\subsubsection{Uppdrag}
Fadderiet ska ansvara för sektionens mottagning av de nya studenterna under
mottagningsperioden. Fadderiet ska tillse att varje ny student har en fadder och ett meningsfullt
nolleuppdrag under mottagningsperioden. Alla aktiviteter för de nya studenterna ska bygga på
frivillighet.
\subsubsection{Organisation}
\paragraph{Fadderigeneral}
Fadderigeneralen leder fadderiets arbete och ansvarar för fadderiet inför sektionsstyrelsen.
\paragraph{Fadderikassör}
Se \S \ref{fadderikassor}.
\paragraph{Utskottet}

Utskottet skall i första hand bestå av sektionsmedlemmar. Finns anledning att rekrytera icke-
sektionsmedlem skall valet motiveras inför sektionsstyrelsen. Utskottets medlemmar med
undantag för fadderikassören utses av sektionsstyrelsen på förslag av fadderigeneralen.
\subsection{Damföreningen}
\label{damforeningen}
\subsubsection{Uppdrag}
Damföreningen är en organisation för sina medlemmar vid Datateknologsektionen i Linköping.
Organisationen skall främja sammanhållningen mellan medlemmarna i damöreningen genom
diverse aktiviteter. Damföreningen ska även aktivt verka för att få fler potentiella medlemmar
att söka sig till de utbildningar som sektionen representerar.

\subsubsection{Organisation}
\paragraph{Ordförande}
Damföreningens ordförande leder och organiserar damföreningens arbete.
\paragraph{Kassör}
Se \S \ref{damforeningskassor}.
\paragraph{Utskottet}

Utskottet skall i första hand bestå av sektionsmedlemmar. Finns anledning att rekrytera icke-
sektionsmedlem skall valet motiveras inför sektionsstyrelsen. Utskottets medlemmar med 
undantag för damföreningens kassör utses av sektionsstyrelsen på förslag av damföreningens
ordförande.
\paragraph{Medlemmar}
Medlem av damföreningen måste anse sig vara av kvinnligt kön eller ickebinär.
\subsubsection{Aktiviteter}
\begin{itemize}
  \item Under året skall den officiella Donnamiddagen hållas.
  \item Kring mottagningen ska damföreningen ordna en aktivitet för de nya medlemmarna i damföreningen så att dessa blir introducerade så snart som möjligt.
\end{itemize}

\subsection{Webb-/infoutskottet}
\label{webbinfo}
\subsubsection{Uppdrag}
Webb-/infoutskottets uppgift är att på olika sätt informera medlemmar om sektionen samt
underhålla och driva sektionens webbsidor. En god kontakt med sektionens medlemmar ska
även upprätthållas och utskottet ska vara lyhörda för deras önskemål.

Webb-/infoutskottet ansvarar även för att:
\begin{itemize}
  \item Bistå sektionen med möjligheten att dokumentera dess aktiviteter.
  \item Information sprids till medlemmarna via regelbundna infomail.
  \item Handha och ansvara för sektionens medlemsregister.
  \item Få in feedback från medlemmarna.
\end{itemize}

\subsubsection{Projekt}
Infoutskottet ska under året utföra följande projekt:
\begin{itemize}
  \item Producera en fotokatalog som läggs upp på sektionens hemsida.
\end{itemize}
Infoutskottet bör under året utföra följande projekt:
\begin{itemize}
  \item Producera sektionens sidor i LiTHanian.
\end{itemize}

\subsubsection{Organisation}
\paragraph{Ordförande}
Ordföranden i Webb-/infoutskottet har till uppgift att organisera och leda utskottets arbete.

Vidare har ordföranden huvudansvaret för information inom sektionen.
\paragraph{Utskottet}

Utskottet skall i första hand bestå av sektionsmedlemmar. Finns anledning att rekrytera icke-
sektionsmedlem skall valet motiveras inför sektionsstyrelsen. Utskottet utses av Webb-/infoutskottets ordförande.

I Webb-/infoutskottet ska följande ansvarsposter tillsättas:
\begin{itemize}
  \item Infomail
  \item Webb
\end{itemize}
I Webb-/infoutskottet bör följande ansvarsposter tillsättas:
\begin{itemize}
  \item Utlandsinformation
  \item Foto
  \item LiTHanian
\end{itemize}

\subsection{Marknadsföringsutskottet}
\label{mafu}
\subsubsection{Uppdrag}
Marknadsföring mot gymnasieskolor ska bedrivas för alla de utbildningsprogram
sektionen representerar.

Marknadsföringsutskottet ska ha samarbete med åtminstone Avdelningen för externa relationer,
Tekniska fakultetskansliet vid Linköpings universitet och Programnämnden för data- och
medieteknik.

Marknadsföringsutskottet ansvarar även för att:
\begin{itemize}
  \item Anordna inspirationsdagar för gymnasieelever.
  \item Sprida information om missionering till sektionens medlemmar.
  \item Rekrytera sektionsmedlemmar till rekryteringsevent.
\end{itemize}
\subsubsection{Organisation}
\paragraph{Ordförande}
Marknadsföringsutskottets arbete leds av utskottets ordförande som – i samarbete med
Programnämnden för data- och medieteknik – är ansvarig för marknadsföring mot
gymnasieskolor av de utbildningar sektionen representerar. Ordförande är även
ansvarig för uppdatering av sektionens och Programnämnden för data- och medietekniks
gymnasieinformation på webben, samt representerar Dateteknologsektionen i LinTeks Marknadsföringsutskott.

\paragraph{Utskottet}

Utskottet skall i första hand bestå av sektionsmedlemmar. Finns anledning att rekrytera icke-
sektionsmedlem skall valet motiveras inför sektionsstyrelsen. Utskottet utses av
marknadsföringsutskottets ordförande.

\subsection{Näringslivsutskottet}
\label{naru}
\subsubsection{Uppdrag}
Näringslivsutskottet ansvarar för att:
\begin{itemize}
  \item Bedriva marknadsföring av sektionen mot företag.
  \item Få in sponsorpengar på minst så mycket som står i budgeten.
  \item Arkivera sponsoravtal och se till att de följs.
  \item Bedriva caseverksamhet.
\end{itemize}
\subsubsection{Organisation}
\paragraph{Ordförande}
Ordföranden i näringslivsutskottet leder utskottets arbete, samt ansvarar för att sektionens
avtal upprättade mellan sektionen och företag efterlevs.
\paragraph{Utskottet}

Utskottet skall i första hand bestå av sektionsmedlemmar. Finns anledning att rekrytera icke-
sektionsmedlem skall valet motiveras inför sektionsstyrelsen. Utskottet utses av
näringslivsutskottets ordförande.

\subsection{Werkmästeriet}
\label{werk}
\subsubsection{Uppdrag}
Det åligger werkmästeriet att:
\begin{itemize}
  \item Handha sektionens egendomar och tillse att nödvändigt reparations- och underhållsarbete utförs.
  \item Ansvara för sektionslokalerna.
  \item Ordna mat till sektionsmötena.
  \item Handha sektionens medaljer.
  \item Ansvara för sektionens märken i märkesbacken.
  \item Ansvara för sektionens väggmålningar på C-huset.
  \item Ansvara för underhåll av sektionens maskotar på Corson.
  \item Ta fram nya och handha gamla sektionsprylar.
\end{itemize}
\subsubsection{Organisation}
\paragraph{Ordförande}
Intendenten leder werkmästeriets arbete.

\paragraph{Utskottet}

Utskottet skall i första hand bestå av sektionsmedlemmar. Finns anledning att rekrytera icke-
sektionsmedlem skall valet motiveras inför sektionsstyrelsen. Utskottet utses av intendenten.

I werkmästeriet ska följande ansvarsposter tillsättas:
\begin{itemize}
  \item Bilansvarig
\end{itemize}

\subsection{Aktivitetsutskottet}
\label{aktu}
\subsubsection{Uppdrag}
Utskottet har som ekonomisk målsättning att efter avslutat år ha nollresultat.

Aktivitetsutskottet ansvarar för att:
\begin{itemize}
  \item Det finns aktiviteter som passar för alla oberoende av fysisk kapacitet eller tidigare erfarenheter.
  \item Utskottet ska verka för ökad sammanhållning mellan sektionens medlemmar.
  \item Utskottet ska uppmuntra och stödja medlemmar som vill göra egna aktiviteter.
\end{itemize}
\subsubsection{Projekt}
Aktivtetutskottet ska under året genomföra följande projekt:
\begin{itemize}
  \item D-Lan
\end{itemize}
\subsubsection{Organisation}
\paragraph{Aktivitetshanteraren}
Aktivitetshanteraren samordnar aktivitetsutskottets arbete och ansvarar för undergrupperna.
\paragraph{Utskottet}

Utskottet skall i första hand bestå av sektionsmedlemmar. Finns anledning att rekrytera icke-
sektionsmedlem skall valet motiveras inför sektionsstyrelsen.

I aktivitetsutskottet ska följande gruppledare utses av sektionsstyrelsen på förslag
av aktivitetshanteraren:
\begin{itemize}
  \item Aktivitetsgruppens ordförande
  \item Projektledare för D-Lan
\end{itemize}

\paragraph{Kassör}
Se \S \ref{aktukassor}.

\paragraph{Undergrupper}
Aktivitetsutskottet består av följande undergrupper:
\begin{itemize}
  \item Aktivitetsgruppen
  \item Projektgruppen för D-Lan
\end{itemize}
Medlemmar i undergrupperna utses av aktivitetshanteraren på förslag av respektive gruppledare.

\subsection{Alumniutskottet}
\label{alumni}
\subsubsection{Uppdrag}
Alumniutskottets uppgift är att på olika sätt främja kontakten mellan
sektionen och de studenter som har avslutat sina studier.

Det åligger Aumniutskottet att:
\begin{itemize}
  \item Arbeta för att främja kontakten med sektionens alumnimedlemmar.
  \item Sköta sektionens alumniweb.
\end{itemize}
\subsubsection{Organisation}
\paragraph{Ordförande}
Ordföranden i Alumniutskottet har till uppgift att leda och organisera utskottets arbete.
\paragraph{Utskottet}
Utskottet utses av Alumniutskottets ordförande. Utskottet skall i första hand bestå av
sektionsmedlemmar. Finns anledning att rekrytera icke-sektionsmedlem skall valet
motiveras inför sektionsstyrelsen.
\subsection{Projektgrupp för Arbetsmarknadsgruppen}
\label{amg}
\subsubsection{Uppdrag}
Arbetsmarknadsgruppen har till uppgift att anordna arbetsmarknadsdagar för studenter inom
program för data och IT.
\subsubsection{Organisation}
\paragraph{Projektledaren för Arbetsmarknadsgruppen}
Projektledaren för arbetsmarknadsgrupppen har till uppgift att leda och organisera
utskottets arbete.
\paragraph{Kassör}
Se \S \ref{amgkassor}.
\paragraph{Utskottet}
Utskottet med undantag för kassören utses av projektledaren för arbetsmarknadsgruppen.
\subsubsection{Val av projektledare och kassör}
Projektledare och kassör för arbetsmarknadsgruppen väljs i samarbete med Y-sektionen vid
Linköpings universitet.

\subsection{Pubutskottet}
\label{pubu}
\subsubsection{Uppdrag}
Utskottet ska anordna pubar för sektionens medlemmar.
\subsubsection{Organisation}
\paragraph{Ordförande}
Ordföranden i pubutskottet har till uppgift att leda och organisera utskottets arbete.
\paragraph{Utskottet}
Utskottet utses av pubutskottets ordförande. Utskottet skall i första hand bestå av
sektionsmedlemmar. Finns anledning att rekrytera icke-sektionsmedlem skall valet
motiveras inför sektionsstyrelsen.
\section{Medaljer och intyg}
\subsection{Vilka som kan få medalj}
Personer som arbetat på ett bra sätt för sektionen kan som bevis på uppskattning tilldelas medalj.
\subsection{Vilka som kan få intyg}
Personer som arbetat på ett bra sätt för sektionen kan som bevis på uppskattning tilldelas diplom.
\subsection{Beslut om utdelningen}
Sektionsmötet beslutar om utdelning av guldmedalj (se stadgarna). Om övriga valörer och diplom
beslutar sektionsstyrelsen.
\subsection{Tidpunkt för utdelning}
Utdelningen ska i första hand ske vid ett gemensamt tillfälle för alla som ska tilldelas medalj och
diplom.
\subsection{Valörer}
För att det inte ska skilja från år till år vilken medalj en viss befattning resulterar i ska följande
valörer vara normgivande. Om man väljer att tilldela en person en medalj av högre valör än
rekommenderat ska detta särskilt motiveras skriftligen.
\begin{labeling}{Förtjänstmedalj}
  \item [Guld] Se stadgar
  \item [Silver] Medlem av sektionsstyrelsen\\*
  Ordförande för festeriet\\*
  Kassör för festeriet\\*
  Ordförande för fadderiet\\*
  Kassör för fadderiet\\*
  Ordförande för damföreningen\\*
  Kassör för damföreningen\\*
  Projektledare för arbetsmarknadsgruppen\\*
  Kassör för arbetsmarknadsgruppen
  \item [Brons] Ordinarie medlemmar av festeriet\\*
  Ordinarie medlemmar av fadderiet\\*
  Ordinarie medlemmar av damföreningens styrelse\\*
  Övriga förtroendevalda
  \item [Förtjänstmedalj] Övriga sektionsaktiva
\end{labeling}


I övrigt gäller att silvermedalj kan tilldelas den som på ett förtjänstfullt sätt arbetat för och med
sektionen. Bronsmedalj kan tilldelas den som jobbat under längre tid för sektionens syften och
förtjänstmedalj kan tilldelas den som under kortare tid jobbat för sektionens syften eller
förtjänstfullt representerat sektionen.
\section{Valberedningen}
\label{val}
Förutom vad som specificeras i stadgarna åligger det Valberedningen att:
\begin{itemize}
  \item Jobba aktivt hela året med att söka kandidater.
  \item Under året söka kandidater till samtliga förtroendeposter i sektionen förutom valberedningens ordförande.
  \item För sektionsstyrelsen, minst två månader inför det aktuella mötet, presentera en plan hur kandidatsökandet ska gå till.
  \item Senast 20 läsdagar före ett val ska ske informera medlemmarna om vilka poster som går att söka och hur man går tillväga för att göra detta.
  \item Innan nomineringarna upprättas intervjua samtliga kandidater, samt låta dessa inkomma med skriftlig komplettering till intervjun om de så önskar.
  \item Inför sektionsmötet motivera sina nomineringar.
\end{itemize}
\section{Tystnadsplikt i valprocesser}
Namn på kandidater som ej officiellt nomineras ska ej diskuteras med personer utanför den grupp
som utför valprocessen. Information som framkommit under intervju skall behandlas med respekt.
Tystnadsplikten gäller även efter uppdraget avslutats.
\subsection{Berörda poster}
Tystnadsplikt gäller för förtroendevalda samt medlemmar i valberedningen.
\section{Högtidlighållande av sektionens bildande}
\subsection{Datateknologsektionens bildande}
Datateknologsektionen bildades 27 januari 1976, d.v.s. året efter att D-programmet bildades.

\subsection{Högtidlighållande}
Det är Datateknologsektionens bildande, inte något av de program som sektionen representerars
bildande som ska högtidlighållas.\\*
Datateknologsektionens bildande bör firas minst var 5:e år. Lämpliga årtal för jubileum ges av \(1976
+ 5k\) eller \(1976 + 2^k\), \(k \in \mathbb{N}\).
\subsection{Festgeneral}
\label{hogtid}
Det åligger vårmötet att välja festgeneral året innan högtidlighållandet äger rum.\\*
Det åligger valberedningen att se till att det finns valbara kandidater till vårmötet.


\end{document}