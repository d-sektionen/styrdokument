\documentclass{datateknologsektionen-document}

\title{Stadgar för Datateknologsektionen}
\date{2021-04-11}

\begin{document}
\hspace{0pt}
\vfill
\begin{center}
  \Huge\textbf{ Stadgar}

  \huge Datateknologsektionen

  \large
  Organisationsnummer: 822002-1409

  \vspace{4mm}
  Antagna 2021-04-11

\end{center}
\vfill
\hspace{0pt}
\pagebreak

\section*{Versionshistorik}
\begin{footnotesize}
  \begin{longtabu} to \linewidth { |l|X| }
    \hline
    \textbf{Sektionsmöte} & \textbf{Ändring} \\ \hline
    98-04-07 & Första versionen  \\ \hline
    98-04-12 & Infört nuvarande stadgar \\ \hline
    99-01-31 & Reviderat \\ \hline
    99-02-22 & Reviderat \\ \hline
    01-03-12 & Reviderat \\ \hline
    01-05-09 & Reviderat \\ \hline
    02-11-05 & Reviderat \\ \hline
    03-11-04 & Reviderat \\ \hline
    06-01-22 & Reviderat \\ \hline
    07-06-24 & Reviderat \\ \hline
    09-06-05 & Reviderat \\ \hline
    09-05-01 & Reviderat \\ \hline
    11-04-14 & Reviderat Enligt proposition ang. vintermöte \\ \hline
    12-05-23 & Reviderat \\ \hline
    12-11-19 & Reviderat \\ \hline
    13-01-24 & Reviderat enligt proposition angående utöka antalet medlemmar \\ \hline
    13-03-17 & Reviderat Enligt proposition angående städning samt proposition angående utbildningsfrågor \\ \hline
    14-10-25 & Reviderat så proposition angående utöka antalet medlemmar är korrekt tillagd \\ \hline
    14-10-25 & Reviderat enligt ``Namnbyte IT Dagarna'' och ``Uppdatering stadgar'' \\ \hline
    15-03-20 & Reviderat enligt ``Flytt av val av tjejföreningens ordförande och kassör'' + Uppdaterat dokumentet till att följa sektionens nya Grafiska profil \\ \hline
    15-06-06 & Reviderat enligt ``Proposition angående val av AMO'' \\ \hline
    16-01-18 & Reviderat enligt ``Motion angående ändring av D-sektionens stadgar'' samt åtgärdat tryckfel  \\ \hline
    16-03-30 & Reviderat enligt ``Proposition angående Arbetsmarknadsgruppens ansvarsfrihet'' \\ \hline
    16-11-24 & Reviderat enligt ``Motion angående byte av föreningsnamn tjejförening till damförening  \\ \hline
    19-01-10 & Reviderat enligt ``Motion angående mastersnordf'', ``Motion angånede namnbyte av nolleperioden'' och ``Motion angående Styrelsens sammansättning och arbetsordning''. Ändrat till latex.  \\ \hline
    19-02-17 & Reviderat enligt ``Proposition angående omstrukturering av §5.13 Höstmötet'', ``Proposition angående aktivitetsutskottets ansvarsfrihet'' och ``Proposition angående Arbetsmarknadsgruppen, LINK''. \\ \hline
    19-12-27 & Reviderat enligt ``Motion angående inval av Aktivitetsutskottets förtroendevalda''.  \\ \hline
    20-02-04 & Reviderat enligt ``Motion angående korrigering av stadgar'' och ``Motion angående uppdelning av webb-/infoutskottet, stadgeändring''.  \\ \hline
    20-05-04 & Reviderat enligt ``Motion angående ändring av Valberedningens organisation'' och ``Proposition angående deadline för inskickning av propositioner och motioner''.  \\ \hline
    21-02-05 & Reviderat enligt ``Proposition angående eventuellt överskott inom utskott''.  \\ \hline
    21-04-11 & Reviderat enligt ``Proposition angående revidering av styrdokument''.  \\ \hline
  \end{longtabu}
\end{footnotesize}

\pagebreak
\section*{Tidigare ändringshistorik}
\begin{labeling}{Med ändringar av}
  \item [Antagna] 1976-01-27
  \item [Med ändringar av]
  1976-11-15 \\*
  1977-10-25 \\*
  1978-12-07 \\*
  1979-05-10 \\*
  1979-11-13 \\*
  1981-05-12 \\*
  1982-05-17 \\*
  1982-11-16 \\*
  1983-02-28 \\*
  1983-05-02 \\*
  1984-01-28 \\*
  1986-02-12 \\*
  1987-04-28 \\*
  1988-05-09 \\*
  1989-09-26 \\*
  1990-10-15 \\*
  1991-03-05 \\*
  1993-04-06 \\*
  1993-11-08 \\*
  1994-04-21 \\*
  1994-11-24 \\*
  1995-11-16 \\*
  1996-04-24 \\*
  1996-11-26 \\*
  1997-04-23 \\*
\end{labeling}
\pagebreak

\tableofcontents

\pagebreak

\section*{Definitioner}
\begin{description}
  \item [Läsdagar]
        Helgfria vardagar under terminstid, dock undantaget ledighet som
        framgår av kalendern för Tekniska högskolan vid Linköpings universitet.
\end{description}

\section{Ändamål}
\subsection{Syfte}
\label{syfte}
Datateknologsektionen har till uppgift att bevaka och tillvarata medlemmarnas intressen
och sprida information om utbildningsprogrammen internt såväl som externt samt att
verka för sammanhållningen mellan sina medlemmar.

Sektionens ekonomiska målsättning är att uppnå ett nollresultat på lång sikt.

\subsection{Utbildningsprogram}
\label{utbildningsprogram}
Sektionen representerar följande program vid Linköpings universitet:
\begin{itemize}
  \item Civilingenjörsprogrammet i Datateknik
  \item Civilingenjörsprogrammet i Informationsteknologi
  \item Civilingenjörsprogrammet i Mjukvaruteknik
  \item Kandidatprogrammet i Innovativ Programmering
  \item Programmen i Datavetenskap
\end{itemize}

\subsection{Namn}
Föreningens namn är Datateknologsektionen.

\subsection{Säte}
Datateknologsektionen har sitt säte i Linköping.

\section{Medlemskap}
\label{medlemskap}
Medlemskap erhålles genom inskrivning i sektionens medlemsregister.

\subsection{Studerandemedlem}
En person äger rätt till studerandemedlemskap så länge man:
\begin{itemize}
  \item studerar eller innehar studieuppehåll i max 2 sammanhängande år på något av de
        utbildningsprogram som sektionen representerar
  \item eller är masterstudent som tidigare tagit ut en kandidatexamen från något av de
        utbildningsprogram som sektionen representerar
\end{itemize}

\subsection{Hedersmedlem}
Till hedersmedlem kan den person väljas som genom extraordinära insatser synnerligen
främjat sektionens intressen och strävanden. Hedersmedlemmar utses av sektionsmötet.

\subsection{Alumnimedlem}
Alumnimedlem är den som under sin studietid varit studerandemedlem i
Datateknologsektionen och tagit ut en examen på något av de program som sektionen
representerar. Alumnimedlem erhåller rätt att delta i sektionens alumniverksamhet.

\subsection{Utträde}
Sektionsmedlem som ej längre önskar vara medlem i sektionen kan begära utträde.
Detta sker skriftligen till sektionsstyrelsen som skall behandla frågan skyndsamt.

\subsection{Uteslutning}
Sektionsmötet kan utesluta en medlem om särskilda skäl föreligger.

\section{Styrdokument}
Sektionens verksamhet regleras, förutom av dessa stadgar, också av reglementet.
Reglementet kan ändras av sektionsmötet.

\section{Verksamhet}
\subsection{Organisation}
\begin{itemize}
  \item Sektionsmötet
  \item Sektionsstyrelsen
        \begin{itemize}
          \item Presidiet
        \end{itemize}
  \item Valberedningen
  \item Revisorer
  \item Utskott
\end{itemize}

\subsection{Utskott}
\label{utskott}
\begin{itemize}
  \item Damföreningen Donna
  \item Fadderiet Staben
  \item Festeriet D-Group
\end{itemize}
Övriga utskott regleras av reglementet.

\subsection{Verksamhetsår}
Sektionens verksamhetsår sträcker sig från och med 1 juli till och med 30 juni följande
år.

\section{Sektionsmöte}
\subsection{Syfte}
Sektionsmötet är sektionens högsta beslutande organ.

\subsection{Verksamhet}
Sektionsmötet har att verkställa erforderliga val, granska styrelsens och övriga organs
verksamhet, dra upp riktlinjer för verksamheten i stort samt fastställa den ekonomiska
ramen för verksamheten.

\subsection{Sammanträden}
Sektionsmötet skall under verksamhetsåret sammanträda vid minst tre ordinarie tillfällen;
ett höstmöte under höstterminen, ett vintermöte i början av vårterminen samt ett vårmöte
senare under vårterminen.

\subsection{Kallelse}
Kallelse och föredragningslista till sektionsmöteskall anslås av sektionsstyrelsen i
sektionens alla rum (ej förråd), samt på sektionens hemsida senast tio läsdagar före
sektionsmötet. Kallelsen skall uppta förslag till föredragningslista. Slutgiltig
föredragningslista och samtliga till mötet hörande handlingar skall anslås senast två
läsdagar före den dag mötet äger rum.

\subsection{Propositioner och motioner}
Motioner måste skriftligen inkomma till sektionsstyrelsen senast sju läsdagar före
sektionsmötet. Propositioner och motioner måste publiceras av sektionsstyrelsen
senast fem läsdagar före sektionsmötet.

\subsection{Extra sektionsmöte}
Extra sektionsmöte skall hållas inom 20 läsdagar efter det att yrkande därom inkommit
till sektionsstyrelsen, samt utlysas i vederbörlig ordning.

Rätt att hos sektionsstyrelsen begära utlysande av extra sektionsmöte tillkommer:
\begin{itemize}
  \item Sektionsstyrelseledamot
  \item Valberedningens ordförande
  \item Revisorerna
  \item En sammanslutning av minst 25 studerandemedlemmar
\end{itemize}

\subsection{Deltagarnas rättigheter}
Närvaro-, yttrande-, förslags- och rösträtt vid sektionsmötet tillkommer varje
studerandemedlem. Närvaro-, yttrande- och förslagsrätt tillkommer varje annan
sektionsmedlem. Röstning med fullmakt får ej förekomma. Sektionsmötet har rätt att
adjungera personer. Med adjungering avses närvaro-, yttrande- och förslagsrätt.
Adjungering medför ej rätt att deltaga i beslut, ej heller medansvar för fattade beslut.

\subsection{Beslutsmässighet}
Sektionsmötet är beslutsmässigt då minst 25 studerandemedlemmar är närvarande.

\subsection{Omröstning}
Beslut fattas med enkel majoritet. Vid lika röstetal har mötesordföranden utslagsröst,
förutom vid personval då valet avgörs med lottning. För att föra in ett nytt ärende på
föredragningslistan erfordras 4/5 majoritet. Under punkten ``övriga frågor'' får ej
behandlas frågor som gäller kostnader för sektionen på högre belopp än 1/5 prisbasbelopp.

Vid val där fler än en person kan bli vald till en post skall antalet valbara kandidater
fastställas innan valet genomförs. Därefter sker omröstning då exakt det fastställda
antalet namn anges för att rösten ska vara giltig. I händelse av att två eller fler
kandidater erhåller lika röstetal och endast en kan bli vald går frågan vidare till
enskild omröstning mellan dessa kandidater. Om lika röstetal erhålls i enskild
omröstning av samtliga kvarvarande kandidater avgörs valet genom lotten.

Förtroendevald får ej deltaga i beslut om egen ansvarsfrihet.

Justeringspersonerna (se \S \ref{justeringavprotokoll}) agerar rösträknare vid omröstning.

\subsection{Nomineringar}
Valberedningen nominerar förslag på kandidater till de poster som skall väljas in,
med undantag för kandidater till valberedningen vilka nomineras av sektionsstyrelsen.

Varje sektionsmedlem äger rätt att väcka motnominering eller motkandidera till
valberedningens och sektionsstyrelsens förslag. Motnomineringar och motkandidaturer
skall anslås i anslutning till valberedningens förslag.

På sektionsmötet skall alla kandidater till en post ha möjlighet att presentera sig.

\subsection{Protokoll}
Vid sektionsmötet ska beslutsprotokoll föras som inom 10 läsdagar skall färdigställas
och upprättats i minst två exemplar, varav ett skall anslås på sektionens hemsida och ett
skall arkiveras.

\subsection{Justering av protokoll}
\label{justeringavprotokoll}
Sektionsmötets protokoll skall justeras av mötesordföranden, mötessekreteraren och två
vid mötet särskilt utsedda ledamöter.

\subsection{Mötesärenden}
Det åligger sektionsmötet att under verksamhetsåret, för nästkommande verksamhetsår:
\begin{itemize}
  \item Välja sektionsstyrelse
  \item Välja ordförande och kassör för i \S \ref{utskott} angivna utskott
  \item Välja förtroendeposter enligt reglementet
  \item Välja valberedning
  \item Välja revisorer
  \item Fastställa sektionens rambudget och sektionsavgift
  \item Behandla verksamhetsberättelse, ekonomisk berättelse och revisionsberättelse för föregående års verksamhet
  \item Besluta om ansvarsfrihet för föregående års verksamhet
\end{itemize}

\section{Sektionsstyrelsen}
\subsection{Organisation}
Styrelsen består minimalt av följande ordinarie ledamöter:
\begin{itemize}
  \item Ordförande
  \item Vice ordförande
  \item Sekreterare
  \item Kassör
\end{itemize}

Övriga styrelseledamöter specificeras i reglementet. Både ordinarie och övriga ledamöter
har samma rättigheter och skyldigheter.

Sektionsstyrelsen beslutar i frågor som innebär en kostnad utanför budget på upp till 2/3 prisbasbelopp.
Sektionsstyrelsen äger rätt att besluta om budgeten under innevarande verksamhetsår under förutsättning
att det inte ändrar det förväntade resultatet med mer än ett prisbasbelopp, samt med krav att
återrapportera beslutet till sektionsmötet.

\subsection{Åtagande}
Sektionsstyrelsen handhar den direkta ledningen av sektionen i enlighet med dessa stadgar.

Det åligger styrelsen att:
\begin{itemize}
  \item Besluta om den löpande verksamheten
  \item Bereda ärenden, vilka skall behandlas vid sektionsmötet
  \item Upprätta förslag till föredragningslista för sektionsmötet
  \item Inför sektionsmötet ansvara för sektionens verksamhet
  \item Verkställa av sektionsmötet fattade beslut
  \item Bevaka medlemmarnas intressen
  \item Hålla sig informerad om verksamheten i sektionens utskott
  \item Bereda val till valberedning och revisorer
  \item Besluta om utbetalning av revisorsarvode efter att revisorerna presenterat
        revisionsberättelse för sektionen och dess utskott
\end{itemize}

\subsection{Sammanträde}
\label{sektionsstyrelsensammantrade}
Rätt att hos ordförande begära utlysande av sektionsstyrelsemöte tillkommer varje
styrelsemedlem samt revisorer. Styrelsen skall hålla minst två protokollförda sammanträden per termin.
Revisorerna äger närvaro-, yttrande- och förslagsrätt på styrelsens sammanträden.

\subsection{Beslutsmässighet}
Styrelsen är beslutsmässig om mer än hälften av ledamöterna är närvarande. Alla
närvarande ledamöter äger rösträtt vid styrelsesammanträden.

\subsection{Entledigande}
Då särskilda skäl föreligger kan styrelsen efter skriftlig ansökan från förtroendevald
entlediga vederbörande samt tillförordna annan person att fullgöra den entledigades
uppgifter till nästa sektionsmöte, då val skall ske. Sektionsmötet äger rätt att entlediga
samtliga förtroendevalda av sektionsmötet.

Styrelsen äger ej rätt att entlediga:
\begin{itemize}
  \item ordförande
  \item vice ordförande
  \item kassör
  \item valberedningens ordförande
  \item revisorerna
\end{itemize}

\subsection{Adjungeringar}
Sektionsstyrelsen liksom sektionsmötet har rätt att adjungera personer. Med adjungering
avses närvaro-, yttrande- och förslagsrätt. Adjungering medför ej rätt att deltaga i beslut,
ej heller medansvar för fattade beslut.

\subsection{Protokoll}
Vid sektionsstyrelsesammanträdena samt presidiesammanträdena med beslut skall protokoll föras.
Inom tio läsdagar skall ett exemplar av protokollet anslås på sektionsspecifik plats, samt
ett arkiveras.

\subsection{Justering av protokoll}
Styrelseprotokollet skall justeras av ordföranden, sekreteraren samt en vid mötet utsedd
justeringsperson.

\subsection{Ordförande}
Det åligger ordföranden att:
\begin{itemize}
  \item leda sektionen och föra sektionens talan
  \item leda sektionsstyrelsens arbete
  \item leda sektionsmöten
\end{itemize}

\subsection{Vice ordförande}
Det åligger vice ordföranden att:
\begin{itemize}
  \item ansvara för samordning av sektionens utskott
\end{itemize}

\subsection{Sekreterare}
Det åligger sekreteraren att:
\begin{itemize}
  \item vid sektionens och styrelsens sammanträden föra protokoll
  \item ansvara för sektionens stadgar
\end{itemize}

\subsection{Kassör}
Det åligger kassören att:
\begin{itemize}
  \item förvalta sektionens ekonomi
  \item ansvara för att inventarie-/lagerföra sektionens tillgångar
  \item vid varje läsperiod eller vid anmodan redovisa sektionens ekonomiska ställning
  \item upprätta budget för nästa verksamhetsår
  \item upprätta bokslut för sektionen och bokföra enligt god bokföringssed
\end{itemize}

\subsection{Presidium}
Presidiet äger rätt att för sektionen besluta i frågor av brådskande karaktär som
innebär en kostnad på upp till 1/10 prisbasbelopp. Presidiet består av styrelsens
ordförande, vice ordförande och kassör.

\subsection{Firmateckning}
Sektionens firma tecknas av ordförande och kassör var för sig.

\section{Valberedningen}
\subsection{Valberedningens organisation}
Valberedningen består av:
\begin{itemize}
  \item Valberedningens ordförande
  \item Minst fyra ledamöter från utbildningar som Datateknologsektionen representerar
\end{itemize}

Valberedningen väljs av sektionsmötet på förslag av sektionsstyrelsen. Valberedningen
svarar inför sektionsmötet.

Sektionsstyrelsen skall sträva efter att:
\begin{itemize}
  \item Ge förslag på ledamöter som representerar flera årskurser
  \item Ge förslag på en ledamot från varje utbildning med nyintag som Datateknologsektionen representerar
\end{itemize}

Finns anledning att inte rekrytera en från vardera program skall valet motiveras inför sektionsmötet.

\subsection{Valberedningens åtaganden}
Det åligger valberedningen att ta fram förslag på kandidater till de poster som väljs
av sektionsmötet, undantaget valberedningen, genom att:

\begin{itemize}
  \item För sektionsstyrelsen, minst två månader inför det aktuella mötet, presentera
        en plan hur kandidatsökandet ska gå till.
  \item Senast 20 läsdagar före ett val ska ske informera medlemmarna om vilka poster
        som går att söka och hur man går tillväga för att göra detta.
  \item Innan nomineringarna upprättas intervjua samtliga kandidater, samt låta dessa
        inkomma med skriftlig komplettering till intervjun om de så önskar.
  \item Inför sektionsmötet motivera sina nomineringar. Om det, enligt valberedningen,
        saknas lämpliga kandidater till en post, åligger det inte valberedningen att
        upprätta nominering för denna post.
\end{itemize}

Alla medlemmar i sektionen äger rätt att inkomma med förslag till valberedningen.
Valberedningens förslag skall anslås senast fem läsdagar före sektionsmötet.

\section{Revision}
\subsection{Revisorer}
Sektionens verksamhet granskas av en sakrevisor och en till två sifferrevisorer, valda
av sektionsmötet på förslag av valberedningen.

Revisor skall vara myndig, får ej vara jävig och ej inneha något annat uppdrag inom sektionen
under verksamhetsåret som granskas.

\subsection{Åligganden}
Det åligger revisorerna att kontinuerligt under verksamhetsåret revidera sektionen och dess utskott,
samt inför avsett sektionsmöte eller senast inom 6 månader efter verksamhetsårets slut avsluta sin
granskning av föregående verksamhetsår och över den företagna revisionen upprätta revisionsberättelse.

Revisorerna skall även innan och under verksamhetsåret informera de förtroendevalda hur revisionen
kommer att genomföras och vilka krav som ställs på bokföring samt verksamhet.

\subsection{Revisionsberättelse}
Revisionsberättelse skall innehålla yttrande i fråga om ansvarsfrihet för berörda
funktionärer.

\subsection{Handlingar}
Räkenskaper och övriga handlingar skall tillställas revisorerna senast 15 läsdagar före
sektionsmötet då ansvarsfrihet behandlas.

\subsection{Avgång}
Om någon befattningshavare inom sektionen avgår, skall granskning av dennes
förvaltning genast verkställas.

\subsection{Rättigheter}
Revisorerna har rätt att närvara vid styrelsemötena (se \S \ref{sektionsstyrelsensammantrade})
samt utskottsmöten. Revisorerna har tillträde till sektionens lokaler och kan anmoda
förtroendevalda att lämna ut information som behövs för en korrekt revision såsom,
men ej begränsat till, protokoll och övriga handlingar.

\section{Stadgar}
\subsection{Stadgeändring}
Förslag till ändringar av dessa stadgar skall skriftligen inlämnas till sektionens styrelse
minst fem läsdagar före det sektionsmöte vid vilket förslaget önskas behandlat.
Ändringsförslaget skall omedelbart anslås av styrelsen.

\subsection{Beslut}
Sektionsmötet skall behandla ändringsförslaget vid två på varandra följande möten med
minst en månads mellanrum.

För beslut fordras att vid båda tillfällena minst 2/3 av antalet närvarande mötesdeltagare
är ense om beslutet.

Vid andra läsningen av stadgeändringen kan det vid förra mötet antagna
stadgeändringsförslaget endast bifallas eller avslås.

\subsection{Tolkning}
Vid frågor om tolkning av dessa stadgar och styrdokument gäller styrelsens mening, tills frågan avgjorts av
sektionsmötet.

\section{Upplösning}
\subsection{Upplösning}
För upplösning av sektionen erfordras att beslut härom fattas av två på varandra följande
sektionsmöten med minst tre månaders mellanrum och på dessa båda sammanträdena
måste minst 9/10 av de närvarande på sektionsmötet godkänna upplösningen.

Utskick till medlemmarna av ett motiverat förslag till beslut om upplösning skall ske
senast tio läsdagar innan båda sektionsmötena där beslut skall fattas.

\subsection{Kvarvarande medel}
Efter att skulder betalats skall kvarvarande tillgångar tillfalla humanitär
hjälporganisation.

\subsection{Likvidation}
När den beslutande omröstningen utfallit så att sektionen skall upplösas, skall
likvidationsförfarandet inledas omedelbart eller den senare dag som sektionsmötet
beslutar.

Sektionsmötet skall utse en eller flera likvidatorer som träder i styrelsens ställe och som
har till uppgift att genomföra likvidationen. Uppdraget att vara revisor upphör inte genom
att föreningen träder i likvidation.


\end{document}