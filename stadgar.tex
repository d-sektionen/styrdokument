\documentclass{datateknologsektionen-document}

\title{Stadgar för Datateknologsektionen}
\date{2019-02-17}

\begin{document}
  \hspace{0pt}
  \vfill
  \begin{center}
    \Huge\textbf{ Stadgar}

    \huge Datateknologsektionen

    \large
    Organisationsnummer: 822002-1409

    Version: 1.06 
    
  \end{center}
  \vfill
  \hspace{0pt}
  \pagebreak

  \section*{Versionshistorik}
  \begin{footnotesize}
    \begin{longtabu} to \linewidth { |l|l|X[2]|p{1.2cm}|p{1.2cm}|X| }
      \hline
      Version & Datum & Anmärkning & Första läsning & Andra läsning & Ansvarig \\ \hline
      0.1 & 1998-04-07 & Första versionen &  &  & Fredrik Claesson \\ \hline
      0.2 & 1998-04-12 & Infört nuvarande stadgar &  &  & Fredrik Claesson \\ \hline
      0.3 & 1999-01-31 & Reviderat &  &  & D-Styret 98/99 Anna Stjerneby \\ \hline
      0.4 & 1999-02-22 & Reviderat &  &  & Sektionsmötet, Anna Stjerneby \\ \hline
      0.5 & 2001-03-12 & Reviderat & 00-11-15 & 01-03-25 & Sektionsmötet, Björn Abrahamsson \\ \hline
      0.6 & 2001-05-09 & Reviderat & 01-05-09 & 01-11-07 & Sektionsmötet, Björn Abrahamsson \\ \hline
      0.7 & 2002-11-05 & Reviderat & 02-05-07 & 02-11-05 & Sektionsmötet, Nicklas Cederholm \\ \hline
      0.8 & 2003-11-04 & Reviderat & 03-04-09 & 03-11-04 & Sektionsmötet, Isak Rydén \\ \hline
      0.9 & 2006-01-22 & Reviderat & 05-04-09 & 05-11-01 & Sektionsmötet, Alexander Sandström Krantz \\ \hline
      0.91 & 2007-06-24 & Reviderat & 06-11-15 & 07-03-28 & Sektionsmötet, Hampus Larsson \\ \hline
      0.92 & 2009-06-05 & Reviderat & 08-11-05 & 09-04-01 & Sektionsmötet, Jakob Fries \\ \hline
      0.93 & 2009-05-01 & Reviderat & 09-11-17 & 10-04-28 & Sektionsmötet, Stefan Johansson \\ \hline
      0.94 & 2011-04-14 & Reviderat Enligt proposition ang. vintermöte &  &  & Sektionsmötet, Arvid Söderström \\ \hline
      0.95 & 2012-05-23 & Reviderat & 12-01-30 & 12-05-02 & Sektionsmötet, Simon Lindén \\ \hline
      0.96 & 2012-11-19 & Reviderat & 12-05-02 & 12-11-06 & Sektionsmötet, Bo Eriksson Nicolin \\ \hline
      0.97 & 2013-01-24 & Reviderat enligt proposition angående utöka antalet medlemmar & 12-11-06 & 12-12-10 & Sektionsmötet, Joy Friberg \\ \hline
      0.98 & 2013-03-17 & Reviderat Enligt proposition angående städning samt proposition angående utbildningsfrågor & 12-12-10 & 13-01-29 & Sektionsmötet, Bo Eriksson Nicolin \\ \hline
      0.99 & 2014-10-25 & Reviderat så proposition angående utöka antalet medlemmar är korrekt tillagd & 12-11-06 & 12-12-10 & Robin Abrahamsson \\ \hline
      0.991 & 2014-10-25 & Reviderat enligt ``Namnbyte IT Dagarna'' och ``Uppdatering stadgar'' & 13-04-24 & 13-11-13 & Sektionsmötet, Robin Abrahamsson \\ \hline
      1.0 & 2015-03-20 & Reviderat enligt ``Flytt av val av tjejföreningens ordförande och kassör'' + Uppdaterat dokumentet till att följa sektionens nya Grafiska profil & 14-11-17 & 15-02-03 & Sektionsmötet, Robin Abrahamsson \\ \hline
      1.01 & 2015-06-06 & Reviderat enligt ``Proposition angående val av AMO'' & 15-02-03 & 15-05-04 & Sektionsmötet, Robin Abrahamsson \\ \hline
      1.02 & 2016-01-18 & Reviderat enligt ``Motion angående ändring av D-sektionens stadgar'' samt åtgärdat tryckfel & 15-05-04 & 15-11-18 & Sektionsmötet, Michael Sörsäter \\ \hline
      1.03 & 2016-03-30 & Reviderat enligt ``Proposition angående Arbetsmarknadsgruppens ansvarsfrihet'' samt `` & 15-11-18 & 16-02-01 & Sektionsmötet, Michael Sörsäter \\ \hline
      1.04 & 2016-11-24 & Reviderat enligt ``Motion angående byte av föreningsnamn tjejförening till damförening & 16-05-02 & 16-11-15 & Sektionsmötet, Mikael Ångman \\ \hline
      1.05 & 2019-01-10 & Reviderat enligt ``Motion angående mastersnordf'', ``Motion angånede namnbyte av nolleperioden'' och ``Motion angående Styrelsens sammansättning och arbetsordning''. Ändrat till latex. & 18-04-24 & 18-11-12 & Sektionsmötet, Emil Nilsson \\ \hline
      1.06 & 2019-02-17 & Reviderat enligt ``Proposition angående omstrukturering av §5.13 Höstmötet'', ``Proposition angående aktivitetsutskottets ansvarsfrihet'' och ``Proposition angående Arbetsmarknadsgruppen, LINK''. & 18-11-12 & 19-02-04 & Sektionsmötet, Emil Nilsson \\ \hline
    \end{longtabu}
  \end{footnotesize}

  \pagebreak
  \section*{Tidigare ändringshistorik}
  \begin{labeling}{Med ändringar av}
    \item [Antagna] 1976-01-27
    \item [Med ändringar av]
      1976-11-15 \\*
      1977-10-25 \\*
      1978-12-07 \\*
      1979-05-10 \\*
      1979-11-13 \\*
      1981-05-12 \\*
      1982-05-17 \\*
      1982-11-16 \\*
      1983-02-28 \\*
      1983-05-02 \\*
      1984-01-28 \\*
      1986-02-12 \\*
      1987-04-28 \\*
      1988-05-09 \\*
      1989-09-26 \\*
      1990-10-15 \\*
      1991-03-05 \\*
      1993-04-06 \\*
      1993-11-08 \\*
      1994-04-21 \\*
      1994-11-24 \\*
      1995-11-16 \\*
      1996-04-24 \\*
      1996-11-26 \\*
      1997-04-23 \\*
  \end{labeling}
  \pagebreak

  \tableofcontents

  \pagebreak

  \section*{Definitioner}
  \begin{description}  
    \item [Läsdagar]
    Helgfria vardagar under terminstid, dock undantaget ledighet som
    framgår av kalendern för Tekniska högskolan vid Linköpings universitet. 
  \end{description}

  \section{Ändamål}
    \subsection{Syfte}
      \label{syfte}
      Datateknologsektionen har till uppgift att bevaka och tillvarata medlemmarnas intressen
      och sprida information om utbildningsprogrammen internt såväl som externt samt att
      verka för sammanhållningen mellan sina medlemmar.
    \subsection{Utbildningsprogram}
      \label{utbildningsprogram}
      De utbildningsprogram som sektionen representerar är:
      \begin{itemize}
        \item Civilingenjörsprogrammet i Datateknik vid Linköpings Tekniska Högskola
        \item Civilingenjörsprogrammet i Informationsteknologi vid Linköpings Tekniska Högskola
        \item Civilingenjörsprogrammet i Mjukvaruteknik vid Linköpings Tekniska Högskola
        \item Kandidatprogrammet i Innovativ Programmering vid Linköpings Tekniska Högskola
        \item Programmen i Datavetenskap vid Linköpings Tekniska Högskola
      \end{itemize}
      
    \subsection{Föreningens namn är Datateknologsektionen}
    \subsection{Datateknologsektionen har sitt säte i Linköping}
  \section{Medlemskap}
    \label{medlemskap}
    Medlemskap erhålles genom inskrivning i sektionens medlemsregister.
    \subsection{Studerandemedlem}
      En person äger rätt till studerandemedlemskap så länge man:
      \begin{itemize}
        \item studerar eller innehar studieuppehåll i max 2 sammanhängande år på något av de
        utbildningsprogram som sektionen representerar (se \S \ref{utbildningsprogram})
        \item eller är masterstudent som tidigare tagit ut en kandidatexamen från något av de
        utbildningsprogram som sektionen representerar (se \S \ref{utbildningsprogram})
      \end{itemize}
    \subsection{Hedersmedlem}
      Till hedersmedlem kan den person väljas som genom extraordinära insatser synnerligen
      främjat sektionens intressen och strävanden. Hedersmedlemmar utses av
      sektionsmötet. (Se vidare \S \ref{medaljer})

    \subsection{Alumnimedlem}
      Alumnimedlem är den som under sin studietid varit studerandemedlem i
      Datateknologsektionen och tagit ut en examen på något av de program som sektionen
      representerar. Alumnimedlem erhåller rätt att delta i sektionens alumniverksamhet.
    \subsection{Utträde}
      Sektionsmedlem som ej längre önskar vara medlem i sektionen kan begära utträde.
      Detta sker skriftligen till sektionsstyrelsen som skall behandla frågan skyndsamt.
    \subsection{Uteslutning}
      Sektionsmötet kan utesluta en medlem om särskilda skäl föreligger.
  \section{Reglemente}
    Sektionens verksamhet regleras i första hand av dessa stadgar men också genom ett
    löpande reglemente nedan kallat ``Löpande reglementet''. Reglemente för kommande
    verksamhetsår antas på vårmötet. Reglementet ska minst omfatta följande punkter:
    \begin{itemize}
      \item Reglemente för sektionsstyrelsen
      \item Reglemente för kassörer
      \item Reglemente för sektionens fonder
      \item Reglemente för sektionsrum
      \item Reglemente för information
      \item Reglemente för studiebevakning
      \item Reglemente för utskott – ett per utskott
      \item Reglemente för medaljer
    \end{itemize}
     
  \section{Verksamhet}
    \subsection{Organisation}
      \begin{itemize}
        \item Sektionsmötet
        \item Valberedningen
        \item Sektionsstyrelsen
        \item Utskottet festeriet
        \item Utskottet fadderiet
        \item Utskottet damföreningen
        \item Revisorer
        \item Presidiet
      \end{itemize}
      Övrigt regleras av reglementet.

    \subsection{Verksamhetsår}
      Sektionens verksamhetsår sträcker sig från och med 1 juli till och med 30 juni följande
      år. Detsamma gäller för alla utskott.
  \section{Sektionsmöte}
    \subsection{Syfte}
      Sektionsmötet är sektionens högsta beslutande organ.
    \subsection{Verksamhet}
      Sektionsmötet har att verkställa erforderliga val, granska styrelsens och övriga organs
      verksamhet, dra upp riktlinjer för verksamheten i stort samt fastställa den ekonomiska
      ramen för verksamheten.
    \subsection{Sammanträden}
      \textit{Sektionsmötet skall under verksamhetsåret sammanträda vid minst tre ordinarie tillfällen;
      ett höstmöte under höstterminen, ett vintermöte i början av vårterminen samt ett vårmöte
      senare under vårterminen.
      }
    \subsection{Kallelse}
      Kallelse och föredragningslista till sektionsmöteskall anslås av sektionsstyrelsen i
      sektionens alla rum (ej förråd), samt på sektionens hemsida senast tio läsdagar före
      sektionsmötet. Kallelsen skall uppta förslag till föredragningslista. Slutgiltig
      föredragningslista och samtliga till mötet hörande handlingar skall anslås senast två
      läsdagar före den dag mötet äger rum.
    \subsection{Propositioner och motioner}
      Propositioner och motioner måste skriftligen inkomma till sektionsstyrelsen senast fem
      läsdagar före sektionsmötet.
    \subsection{Extra sektionsmöte}
      Extra sektionsmöte skall hållas inom 20 läsdagar efter det att yrkande därom inkommit
      till sektionsstyrelsen, samt utlysas i vederbörlig ordning.

      Rätt att hos sektionsstyrelsen begära utlysande av extra sektionsmöte tillkommer:
      \begin{itemize}
        \item Sektionsstyrelseledamot
        \item Valberedningens ordförande
        \item Revisorerna
        \item En sammanslutning av minst 25 studerandemedlemmar
      \end{itemize}
      
    \subsection{Deltagarnas rättigheter}
      Närvaro-, yttrande-, förslags- och rösträtt vid sektionsmötet tillkommer varje
      studerandemedlem (Se \S \ref{medlemskap}). Närvaro-, yttrande- och förslagsrätt tillkommer varje annan
      sektionsmedlem. Röstning med fullmakt får ej förekomma. Sektionsmötet har rätt att
      adjungera personer. Med adjungering avses närvaro-, yttrande- och förslagsrätt.
      Adjungering medför ej rätt att deltaga i beslut, ej heller medansvar för fattade beslut.
    \subsection{Beslutsmässighet}
      Sektionsmötet är beslutsmässigt då minst 25 studerandemedlemmar är närvarande.
    \subsection{Omröstning}
      Beslut fattas med enkel majoritet. Vid lika röstetal har mötesordföranden utslagsröst.
      För att föra in ett nytt ärende på föredragningslistan erfordras 4/5 majoritet. Under
      punkten ``övriga frågor'' får ej behandlas frågor som gäller kostnader för sektionen på
      högre belopp än 1/5 prisbasbelopp.

      Förtroendevald får ej deltaga i beslut om egen ansvarsfrihet.

      Justeringspersonerna agerar rösträknare vid omröstning. (se \S \ref{justeringavprotokoll})
    \subsection{Protokoll}
      Vid sektionsmötet ska beslutsprotokoll föras som inom 10 läsdagar skall färdigställas
      och upprättats i minst två exemplar, varav ett skall anslås på sektionens hemsida och ett
      skall arkiveras.
    \subsection{Justering av protokoll}
      \label{justeringavprotokoll}
      Sektionsmötets protokoll skall justeras av mötesordföranden, mötessekreteraren och två
      vid mötet särskilt utsedda ledamöter.
    \subsection{Vårmötet}
      Det åligger sektionsmötet att på vårmötet, för nästkommande verksamhetsår:
      \begin{itemize}
        \item Välja marknadföringsutskottets ordförande
        \item Välja näringslivsansvarig
        \item Välja studienämndsordförande D
        \item Välja studienämndsordförande IT
        \item Välja studienämndsordförande U
        \item Välja studienämndsordförande IP
        \item Välja studienämndsordförande masternivå
        \item Välja intendent
        \item Välja Webb/-infochef
        \item Välja aktivitetshanterare
        \item Välja revisorer
        \item Välja valberedningens ordförande
        \item Välja alumniutskottets ordförande
        \item Välja aktivitetsutskottets kassör
        \item Välja Pubutskottsordförande
        \item Fastställa sektionens budget och sektionsavgift
        \item Fastställa festeriets budget
        \item Fastställa fadderiets budget
        \item Fastställa damföreningens budget
        \item Fastställa aktivitetsutskottets budget
        \item Fastställa Arbetsmarknadsgruppens budget
        \item Fastställa reglemente
      \end{itemize}
    \subsection{Höstmötet}
      Det åligger sektionsmötet att på höstmötet:
      \begin{itemize}
        \item Välja fadderigeneral för kommande mottagningsperiod
        \item Välja fadderikassör för kommande mottagningsperiod
        \item Behandla verksamhetsberättelse, ekonomisk berättelse och årsrevisionsberättelse för föregående års sektionsstyrelse
        \item Besluta om ansvarsfrihet för föregående års sektionsstyrelse
        \item Behandla verksamhetsberättelse, ekonomisk berättelse och årsrevisionsberättelse för föregående års fadderi
        \item Besluta om ansvarsfrihet för föregående års fadderi
        \item Behandla verksamhetsberättelse, ekonomisk berättelse och årsrevisionsberättelse för föregående års festeri
        \item Besluta om ansvarsfrihet för föregående års festeri
        \item Behandla verksamhetsberättelse, ekonomisk berättelse och årsrevisionsberättelse för föregående års damförening
        \item Besluta om ansvarsfrihet för föregående års damförening
        \item Behandla verksamhetsberättelse, ekonomisk berättelse och årsrevisionsberättelse för föregående års aktivitetsutskott
        \item Besluta om ansvarsfrihet för föregående års aktivitetsutskott
        \item Behandla verksamhetsberättelse, ekonomisk berättelse och årsrevisionsberättelse för föregående års Arbetsmarknadsgruppen
        \item Besluta om ansvarsfrihet för föregående års Arbetsmarknadsgruppen
      \end{itemize}
    \subsection{Vintermötet}
      Det åligger sektionsmötet att, på vintermötet, för nästkommande verksamhetsår:
      \begin{itemize}
        \item Välja sektionsordförande
        \item Välja sektionskassör
        \item Välja vice ordförande
        \item Välja sekreterare
        \item Välja utbildningsutskottets ordförande
        \item Välja arbetsmiljöombud
        \item Välja styrelseledamöter
        \item Välja festeriets ordförande
        \item Välja festeriets kassör
        \item Välja Arbetsmarknadsgruppens ledning enligt Datateknologsektionens reglemente.
        \item Välja damföreningens ordförande
        \item Välja damföreningens kassör 
      \end{itemize}

  \section{Val}
    \subsection{Valberedningens organisation}
      Valberedningen består av:
      \begin{itemize}
        \item Valberedningens ordförande
        \item Minst en ledamot från varje utbildning med nyintag som Datateknologsektionen representerar
      \end{itemize}
      Valberedningens ledamöter väljs av styrelsen på förslag av valberedningens ordförande.
    \subsection{Valberedningens åligganden}
      Det åligger valberedningen att ta fram förslag på kandidater till de poster som väljs av
      sektionsmötet eller sektionsstyrelsen. Om det, enligt valberedningen, saknas lämpliga
      kandidater till en post, åligger det inte valberedningen att upprätta nominering för denna
      post.

      Alla medlemmar i sektionen äger rätt att inkomma med förslag till valberedningen.
      Valberedningens förslag skall anslås senast fem läsdagar före sektionsmötet.
    \subsection{Motnominering}
      Varje sektionsmedlem äger rätt att väcka motnominering till valberedningens förslag.
    \subsection{Presentation}
      På sektionsmötet skall alla kandidater till en post ha möjlighet att presentera sig.
  \section{Sektionsstyrelsen}
    \subsection{Organisation}
      Styrelsen består minimalt av följande ordinarie ledamöter:
      \begin{itemize}
        \item Ordförande
        \item Sekreterare
        \item Kassör
        \item Utbildningsutskottets ordförande
      \end{itemize}
      Övriga styrelseledamöter specificeras i reglementet. Både ordinarie och övriga ledamöter
      har samma rättigheter och skyldigheter.
    \subsection{Syfte}
      Sektionsstyrelsen handhar den direkta ledningen av sektionen i enlighet med sektionens
      syfte. (Se \S \ref{syfte})
    \subsection{Sammanträde}
      \label{sektionsstyrelsensammantrade}
      Rätt att hos ordförande begära utlysande av sektionsstyrelsemöte tillkommer varje
      styrelsemedlem. Styrelsen sammanträder minst fyra gånger per termin. Revisorerna äger
      närvaro-, yttrande- och förslagsrätt på sammanträdena

      Det åligger styrelsen att:
      \begin{itemize}
        \item Besluta om den löpande verksamheten
        \item Bereda ärenden, vilka skall behandlas vid sektionsmötet
        \item Upprätta förslag till föredragningslista för sektionsmötet
        \item Inför sektionsmötet ansvara för sektionens verksamhet
        \item Verkställa av sektionsmötet fattade beslut
        \item Bevaka medlemmarnas intressen
        \item Hålla sig informerad om verksamheten i sektionens utskott
      \end{itemize}
    \subsection{Beslutsmässighet}
      Styrelsen är beslutsmässig om mer än hälften av ledamöterna är närvarande. Alla
      närvarande ledamöter äger rösträtt vid styrelsesammanträden.
    \subsection{Entledigande}
      Då särskilda skäl föreligger kan styrelsen efter skriftlig ansökan från förtroendevald
      entlediga vederbörande samt tillförordna annan person att fullgöra den entledigades
      uppgifter till nästa sektionsmöte, då val skall ske. Sektionsmötet äger rätt att entlediga
      samtliga förtroendevalda av sektionsmötet.

      Styrelsen äger ej rätt att entlediga:
      \begin{itemize}
        \item ordförande
        \item kassör
        \item festeriets ordförande
        \item festeriets kassör
        \item fadderigeneralen
        \item fadderiets kassör
        \item revisorerna
      \end{itemize}
    \subsection{Adjungeringar}
      Sektionsstyrelsen liksom sektionsmötet har rätt att adjungera personer. Med adjungering
      avses närvaro-, yttrande- och förslagsrätt. Adjungering medför ej rätt att deltaga i beslut,
      ej heller medansvar för fattade beslut.
    \subsection{Protokoll}
      Vid sektionsstyrelsesammanträdena samt presidiesammanträdena skall protokoll föras.
      Inom tio läsdagar skall ett exemplar av protokollet anslås på sektionsspecifik plats, samt
      ett arkiveras.
    \subsection{Justering av protokoll}
      Styrelseprotokollet skall justeras av ordföranden, sekreteraren samt en vid mötet utsedd
      justeringsperson.
    \subsection{Ordförande}
      Det åligger ordföranden att
      \begin{itemize}
        \item leda sektionen och föra sektionens talan
      \end{itemize}
      
    \subsection{Sekreterare}
      Det åligger sekreteraren att:
      \begin{itemize}
        \item vid sektionens sammanträden föra protokoll
        \item ansvara för sektionens stadgar
      \end{itemize}
     
    \subsection{Kassör}
      Det åligger kassören att:
      \begin{itemize}
        \item förvalta sektionens ekonomi
        \item upprätta budget för nästa verksamhetsår
        \item upprätta bokslut för sektionen och bokföra enligt god bokföringssed
      \end{itemize}
    \subsection{Presidium}
      Presidiet äger rätt att för sektionen besluta i frågor av brådskande karaktär. Se även
      \S \ref{ekonomisektionen}. Presidiet består av styrelsens ordförande och kassör.
    \subsection{Firmateckning}
      Sektionens firma tecknas av ordförande och kassör var för sig.
  
  \section{Utskott}
    \subsection{Festeriet}
      \subsubsection{Namn}
        Festeriet heter D-Group.
    \subsection{Fadderiet}
      \subsubsection{Namn}
        Fadderiet heter Staben.
    \subsection{Damförening}
      \subsubsection{Namn}
        Damföreningen heter Donna.
    \subsection{Arbetsmarknadsgruppen}
      \subsubsection{Namn}
        Utskottet heter Arbetsmarknadsgruppen.
    \subsection{Aktivitetutskottet}
      \subsubsection{Namn}
        Utskottet heter Aktivitetsutskottet

  \section{Insignia}
    \subsection{Emblem}
      Sektionen representeras med ett officiellt emblem framröstat av sektionsmötet.
    \subsection{Medaljer}
      \label{medaljer}
      Guldmedaljen tilldelas person som under lång tid och på ett beundransvärt sätt arbetat
      för Datateknologsektionen i dess mål och strävanden. Hedersmedlem erhåller
      automatiskt guldmedalj. En person kan bara tilldelas en guldmedalj. Guldmedalj tilldelas
      av sektionsmötet.
      
      Övriga valörer preciseras i reglemente.

  \section{Ekonomi}
    \subsection{Sektionen}
      \label{ekonomisektionen}
      Sektionsstyrelsen beslutar i frågor som innebär en kostnad på upp till 2/3 prisbasbelopp.
      Presidiet äger rätt att besluta i frågor av brådskande karaktär som innebär en kostnad på
      upp till 1/10 prisbasbelopp.
    \subsection{Festeriet}
      Festeriets ekonomiska målsättning är att uppnå ett nollresultat.
    \subsection{Fadderiet}
      Fadderiets ekonomiska målsättning är att uppnå ett nollresultat.
    \subsection{Damföreningen}
      Damföreningens ekonomiska målsättning är att uppnå ett nollresultat.
    \subsection{Arbetsmarknadsgruppen}
      Arbetsmarknadsgruppens ekonomiska målsättning är att uppnå ett vinstresultat.
    \subsection{Aktivitetsutskottet}
      Aktivitetsutskottets ekonomiska målsättning är att uppnå ett nollresultat.

  \section{Revision}
    \subsection{Revisorer}
      Två revisorer, valda av sektionsmötet, ska granska hela sektionens verksamhet.
      Revisorerna ska agera både sak- och sifferrevisor.

      Revisor skall vara myndig och får ej vara jävig.
    \subsection{Åligganden}
      Revisorerna skall före höstmötet avsluta sin granskning av föregående års verksamhet
      och över den företagna revisionen upprätta revisionsberättelse.
    \subsection{Revisionsberättelse}
      Revisionsberättelse skall innehålla yttrande i fråga om ansvarsfrihet för berörda
      funktionärer.
    \subsection{Handlingar}
      Räkenskaper och övriga handlingar skall tillställas revisorerna senast 15 läsdagar före
      höstmötet.
    \subsection{Avgång}
      Om någon befattningshavare inom sektionen avgår, skall granskning av dennes
      förvaltning genast verkställas.
    \subsection{Rättigheter}
      Revisorerna har rätt att närvara vid styrelsemötena (se \S \ref{sektionsstyrelsensammantrade}). Revisorerna har tillträde till
      sektionens lokaler och kan anmoda förtroendevalda att lämna ut information som
      behövs för en korrekt revision.
  \section{Stadgar}
    \subsection{Stadgeändring}
      Förslag till ändringar av dessa stadgar skall skriftligen inlämnas till sektionens styrelse
      minst fem läsdagar före det sektionsmöte vid vilket förslaget önskas behandlat.
      Ändringsförslaget skall omedelbart anslås av styrelsen.
    \subsection{Beslut}
      Sektionsmötet skall behandla ändringsförslaget vid två på varandra följande möten med
      minst en månads mellanrum.

      För beslut fordras att vid båda tillfällena minst 2/3 av antalet närvarande mötesdeltagare
      är ense om beslutet.

      Vid andra läsningen av stadgeändringen kan det vid förra mötet antagna
      stadgeändringsförslaget endast bifallas eller avslås.
    \subsection{Tolkning}
      Vid frågor om tolkning av dessa stadgar gäller styrelsens mening, tills frågan avgjorts av
      sektionsmötet.
  \section{Upplösning}
    \subsection{Upplösning}
      För upplösning av sektionen erfordras att beslut härom fattas av två på varandra följande
      sektionsmöten med minst tre månaders mellanrum och på dessa båda sammanträdena
      måste minst 9/10 av de närvarande på sektionsmötet godkänna upplösningen.

      Utskick till medlemmarna av ett motiverat förslag till beslut om upplösning skall ske
      senast tio läsdagar innan båda sektionsmötena där beslut skall fattas.
    \subsection{Kvarvarande medel}
      Efter att skulder betalats skall kvarvarande tillgångar tillfalla humanitär
      hjälporganisation.
    \subsection{Likvidation}
      När den beslutande omröstningen utfallit så att sektionen skall upplösas, skall
      likvidationsförfarandet inledas omedelbart eller den senare dag som sektionsmötet
      beslutar.

      Sektionsmötet skall utse en eller flera likvidatorer som träder i styrelsens ställe och som
      har till uppgift att genomföra likvidationen. Uppdraget att vara revisor upphör inte genom
      att föreningen träder i likvidation. 


\end{document}